%%%%%%%%%%%%%%%%%%%%%%%%%%%%%%%%%%%%%%%%%%%%%%%%%%%%%%%%%%%%%%%%%%%%%%
%%  Copyright by Wenliang Du.                                       %%
%%  This work is licensed under the Creative Commons                %%
%%  Attribution-NonCommercial-ShareAlike 4.0 International License. %%
%%  To view a copy of this license, visit                           %%
%%  http://creativecommons.org/licenses/by-nc-sa/4.0/.              %%
%%%%%%%%%%%%%%%%%%%%%%%%%%%%%%%%%%%%%%%%%%%%%%%%%%%%%%%%%%%%%%%%%%%%%%

\newcommand{\commonfolder}{../../common-files}

\documentclass[11pt]{article}

\usepackage[most]{tcolorbox}
\usepackage{times}
\usepackage{epsf}
\usepackage{epsfig}
\usepackage{amsmath, alltt, amssymb, xspace}
\usepackage{wrapfig}
\usepackage{fancyhdr}
\usepackage{url}
\usepackage{verbatim}
\usepackage{fancyvrb}
\usepackage{adjustbox}
\usepackage{listings}
\usepackage{color}
\usepackage{subfigure}
\usepackage{cite}
\usepackage{sidecap}
\usepackage{pifont}
\usepackage{mdframed}
\usepackage{textcomp}
\usepackage{enumitem}
\usepackage{hyperref}


% Horizontal alignment
\topmargin      -0.50in  % distance to headers
\oddsidemargin  0.0in
\evensidemargin 0.0in
\textwidth      6.5in
\textheight     8.9in 

\newcommand{\todo}[1]{
\vspace{0.1in}
\fbox{\parbox{6in}{TODO: #1}}
\vspace{0.1in}
}


\newcommand{\unix}{{\tt Unix}\xspace}
\newcommand{\linux}{{\tt Linux}\xspace}
\newcommand{\minix}{{\tt Minix}\xspace}
\newcommand{\ubuntu}{{\tt Ubuntu}\xspace}
\newcommand{\setuid}{{\tt Set-UID}\xspace}
\newcommand{\openssl} {\texttt{openssl}}


\pagestyle{fancy}
\lhead{\bfseries SEED Labs}
\chead{}
\rhead{\small \thepage}
\lfoot{}
\cfoot{}
\rfoot{}


\definecolor{dkgreen}{rgb}{0,0.6,0}
\definecolor{gray}{rgb}{0.5,0.5,0.5}
\definecolor{mauve}{rgb}{0.58,0,0.82}
\definecolor{lightgray}{gray}{0.90}


\lstset{%
  frame=none,
  language=,
  backgroundcolor=\color{lightgray},
  aboveskip=3mm,
  belowskip=3mm,
  showstringspaces=false,
%  columns=flexible,
  basicstyle={\small\ttfamily},
  numbers=none,
  numberstyle=\tiny\color{gray},
  keywordstyle=\color{blue},
  commentstyle=\color{dkgreen},
  stringstyle=\color{mauve},
  breaklines=true,
  breakatwhitespace=true,
  tabsize=3,
  columns=fullflexible,
  keepspaces=true,
  escapeinside={(*@}{@*)}
}

\newcommand{\newnote}[1]{
\vspace{0.1in}
\noindent
\fbox{\parbox{1.0\textwidth}{\textbf{Note:} #1}}
%\vspace{0.1in}
}


%% Submission
\newcommand{\seedsubmission}{
Debe enviar un informe de laboratorio detallado, con capturas de pantalla, para describir lo que ha hecho y lo que ha observado.
También debe proporcionar una explicación a las observaciones que sean interesantes o sorprendentes.
Enumere también los fragmentos de código más importantes seguidos de una explicación. No recibirán créditos aquellos fragmentos de códigos que no sean explicados.}

%% Book
\newcommand{\seedbook}{\textit{Computer \& Internet Security: A Hands-on Approach}, 2nd
Edition, by Wenliang Du. Para más detalles \url{https://www.handsonsecurity.net}.\xspace}

%% Videos
\newcommand{\seedisvideo}{\textit{Internet Security: A Hands-on Approach},
by Wenliang Du. Para más detalles \url{https://www.handsonsecurity.net/video.html}.\xspace}

\newcommand{\seedcsvideo}{\textit{Computer Security: A Hands-on Approach},
by Wenliang Du. Para más detalles \url{https://www.handsonsecurity.net/video.html}.\xspace}

%% Lab Environment
\newcommand{\seedenvironment}{Este laboratorio ha sido testeado en nuestra imagen pre-compilada de una VM con Ubuntu 16.04, que puede ser descargada del sitio oficial de SEED.\xspace}

\newcommand{\seedenvironmentA}{Este laboratorio ha sido testeado en nuestra imagen pre-compilada de una VM con Ubuntu 16.04, que puede ser descargada del sitio oficial de SEED.\xspace}

\newcommand{\seedenvironmentB}{Este laboratorio ha sido testeado en nuestra imagen pre-compilada de una VM con Ubuntu 20.04, que puede ser descargada del sitio oficial de SEED .\xspace}

\newcommand{\seedenvironmentC}{Este laboratorio ha sido testeado en nuestra imagen pre-compilada de una VM con Ubuntu 20.04, que puede ser descargada del sitio oficial de SEED. Sin embargo, la mayoría de nuestros laboratorios pueden ser realizados en la nube para esto Ud. puede leer nuestra guía que explica como crear una VM de SEED en la nube.\xspace}

\newcommand{\seedenvironmentAB}{
Este laboratorio ha sido testeado en nuestras imagenes pre-compiladas de una VM con Ubuntu 16.04 y otra con Ubuntu 20.04, que pueden ser descargadas del sitio oficial de SEED.\xspace}

\newcommand{\nodependency}{Dado que utilizamos contenedores para configurar el entorno de laboratorio, este laboratorio no depende estrictamente de la VM de SEED. Puede hacer este laboratorio utilizando otras máquinas virtuales, máquinas físicas o máquinas virtuales en la nube.\xspace}

\newcommand{\adddns}{You do need to add the required IP address mapping to
the \texttt{/etc/hosts} file.\xspace}






\newcommand{\seedlabcopyright}[1]{
\vspace{0.1in}
\fbox{\parbox{6in}{\small Copyright \copyright\ {#1}\ \ by Wenliang Du.\\
      Este trabajo se encuentra bajo licencia Creative Commons.
       Attribution-NonCommercial-ShareAlike 4.0 International License.
       Si ud. remezcla, transforma y construye a partir de este material,
       Este aviso de derechos de autor debe dejarse intacto o reproducirse de una manera que sea razonable para el medio en el que se vuelve a publicar el trabajo.
       }}
\vspace{0.1in}
}





\newcommand{\firewallFigs}{./Figs}
\lhead{\bfseries SEED Labs -- Laboratorio de Evasión de Firewall}

\begin{document}



\begin{center}
{\LARGE Laboratorio de Evasión de Firewall: Bypasseando Firewalls usando VPN}
\end{center}


\seedlabcopyright{2018}

\newcounter{task}
\setcounter{task}{1}
\newcommand{\tasks} {\bf {\noindent (\arabic{task})} \addtocounter{task}{1} \,}


% *******************************************
% SECTION
% ******************************************* 
\section{Descripción}

Organizaiones, Internet Service Providers (ISPs) y paises a menudo bloquean el acceso a determinados sitios externos a sus usuarios internos. Esto es llamado filtrado de salida o egress filtering.
Por ejemplo, para prevenir distracción en horarios laborales, muchas companias configuran la salida de sus firewalls para bloquear sitios de redes sociales, por lo que sus empleados no podrán acceder a estos dentro de la red interna. Por razones políticas, muchos países configurar filtrados de salida en sus ISPs para bloquear a su gente el acceso a determinados sitios foráneos. Desafortunadamente, estos firewalls pueden ser fácilmente bypasseados y existen servicios/productos que ayudan a los usuarios bypassear estos firewalls, estas soluciones están a la alcance de todos. La tecnología más usada para bypassear estos filtrados de salida son las Virtual Private Network (VPN).
Esta tecnología es ampliamente usada por usuarios que poseen smartphones y que son afectados por este tipo de bloqueo; existen muchas aplicaciones VPN (para Android, iOs y otras plataformas) que ayudan a los usuarios a evadir estas reglas de filtrado que se aplican en los firewalls.

El objetivo de este laboratorio es que los estudiantes vean como funciona una VPN y como una VPN puede ayudar a bypassear los filtrados de salida de un firewall.
En este Laboratorio, implementaremos una VPN muy simple que usaremos para bypassear firewalls. Una VPN típica depende de dos piezas: IP tunneling y el cifrado. El tunneling es esencial ya que será quien nos ayude a byppasear firewalls; el cifrado es para proteger el contenido del tráficos que viaja a través del tunel de la VPN.
Por un tema de simplicidad, en este laboratorio solamente nos centramos en el tunneling, por lo que el tráfico dentro de nuestro tunel no estará cifrado. Tenemos otro laboratorio sobre VPN en donde cubrimos tunneling y cifrado. Si los lectores están interesados, pueden trabajar sobre este laboratorio para aprender como construir una VPN completa. En este Laboratorio solamente nos centramos en como usar un tunel VPN para bypassear firewalls.

Este laboratorio cubre los siguientes tópicos:

\begin{itemize}[noitemsep]
\item Firewall
\item VPN
\end{itemize}


\paragraph{Lectura y Videos.}
Para una cobertura más detallada sobre firewalls, técnicas de evasión y VPN puede consultar:

\begin{itemize}
\item Capítulos 17 y 19 del libro de SEED, \seedbook
\item Secciones 8 y 9 del curso de SEED en Udemy, \seedisvideo
\end{itemize}


\paragraph{Entorno de Laboratorio.} \seedenvironmentB
 


% *******************************************
% SECTION
% ******************************************* 
\section{Tareas del Laboratorio}




% -------------------------------------------
% SUBSECTION
% ------------------------------------------- 
\subsection{Tarea 1: Setup de la Máquina Virtual}

 
We need two machines, one inside the firewall, and the other
outside the firewall. The objective is to help the machine
inside the firewall to reach out to the external sites blocked by the firewall. 
We use two virtual machines, VM1 and VM2, for these two machines. 
VM1 and VM2 are supposed to be two machines connected via the Internet through
routers. This setup may require more than two VMs. 
For the sake of simplicity, we use a LAN to emulate the Internet connection.
Basically, we simply connect VM1 and VM2 to a LAN using the \texttt{NAT Network} 
adapter. Figure~\ref{vpn_firewall:fig:labsetup} depicts the lab setup.

\begin{figure}[htb]
  \begin{center}
    \includegraphics[width=0.9\textwidth]{\firewallFigs/Host2Gateway.pdf}
  \end{center}
  \caption{Lab Environment Setup}
  \label{vpn_firewall:fig:labsetup}
\end{figure}



% -------------------------------------------
% SUBSECTION
% ------------------------------------------- 
\subsection{Tara 2: Setup del Firewall}

In this task, you will set up a firewall on VM1 to block the access of a target website. You
need to make sure that the IP address of the target web site is either fixed or in a fixed
range; otherwise, you may have trouble completely blocking this website. Please refer to the
Firewall Lab for details about how to blocking websites.

In the real world, the firewall should run on a separate machine, not on VM1. To minimize the number of VMs
used in this lab, we put the firewall on VM1. Setting up the firewall on VM1 requires the superuser
privilege, and so does the setup of the VPN tunnel. One may immediately say that if we already
have the superuser privilege, why cannot we just simply disable the firewall on VM1. This is a good argument,
but keep in mind, we put the firewall on VM1 simply because we do not want to create another VM
in the lab environment. Therefore, although you have the superuser privilege on VM1, you are
not allowed to use the privilege to reconfigure the firewall. You have to use VPN to bypass
it.

Compared to putting the firewall on an external machine, putting the firewall on VM1 does have
a small issue that we need to deal with. When we set up the firewall to block packets, we need
to make sure not to block the packets from getting to the virtual interface used by the VPN, or even our
VPN will not be able to get the packets.  Therefore, we cannot set the firewall rule before the
routing, nor can we set the firewall rule on the virtual interface. We just need to set the
rule on VM1's real network interface, so it will not affect the packets that go to the virtual
interface. The following command blocks all traffic to \texttt{93.184.216.0/24} 
network (\texttt{example.com}). 

\begin{lstlisting}
$ sudo iptables -A OUTPUT -o enp0s3 -d 93.184.216.0/24 -j DROP
\end{lstlisting}

Please identify a website that you would like to block, set up the firewall,
and then demonstrate that your firewall is working and the target IP address is no longer 
reachable. Provide screenshots in your lab report.  


% -------------------------------------------
% SUBSECTION
% ------------------------------------------- 
\subsection{Tarea 3: Bypasseando el Firewall usando VPN}



The idea of using VPN to bypass firewall is depicted in 
Figure~\ref{vpn_firewall:fig:bypassing}. 
We establish a VPN tunnel between VM1 (VPN Client VM) 
and VM2 (VPN Server VM). 
When a user on VM1 tries to access a blocked site, the traffic will not directly 
go through its network adapter, because it will be blocked. Instead, the 
packets to the blocked site from VM1 will be routed to the VPN tunnel and arrive at VM2. Once
they arrive there, VM2 will route them to the final destination. 
When the reply packets come back, it will come back to VM2, which will then redirect the
packets to the VPN tunnel, and eventually get the packet back to VM1. That is how the VPN helps
VM1 to bypass firewalls. 

\begin{figure}[htb]
\begin{center}
\includegraphics[width=1.0\textwidth]{\firewallFigs/BypassingFirewall.pdf}
\end{center}
\caption{Bypassing firewall using VPN}
\label{vpn_firewall:fig:bypassing}
\end{figure}
 


We have created a sample VPN program, including a client program (\texttt{vpnclient})  and
a server program (\texttt{vpnserver}), both of which can be downloaded from
this lab's web site. This simple VPN program only establishes a VPN tunnel 
between the client and server; it does not encrypt the tunnel traffic.
The program is explained in details in the SEED book (in the VPN chapter).


\begin{figure}[htb]
\begin{center}
\includegraphics[width=0.9\textwidth]{\firewallFigs/ClientServerTunnel.pdf}
\end{center}
\caption{VPN client and server}
\label{vpn_firewall:fig:client_server}
\end{figure}

The \texttt{vpnclient} and \texttt{vpnserver} programs are the two ends of
a VPN tunnel. They communicate with each other using either TCP or UDP via the sockets
depicted in Figure~\ref{vpn_firewall:fig:client_server}. In our sample code, we choose
to use UDP for the sake of simplicity.  The dotted line between the
client and server depicts the path for the VPN tunnel.
The VPN client and server programs connect to the hosting system via a
TUN interface, through which they do two things: (1) get IP packets from
the hosting system, so the packets can be sent through the tunnel, (2) get IP packets from the
tunnel, and then forward it to the hosting system, which will forward the
packet to its final destination.
The following procedure describes how to create a VPN tunnel
using the \texttt{vpnclient} and \texttt{vpnserver} programs.


\paragraph{Paso 1: Iniciar el Servidor VPN.}
We first run the VPN server program \texttt{vpnserver} on the Server VM.
After the program runs, a virtual TUN network interface will appear
in the system (we can see it using the \texttt{"ifconfig -a"} command; the name of the
interface will be \texttt{tun0} in most cases, but they can be
\texttt{tunX}, where \texttt{X} is a number).
This new interface is not yet configured, so we need to configure it by giving it an IP
address. We use \texttt{192.168.53.1} for this interface, but you can use 
other IP addresses. 


Run the following commands. The first command will start the server
program, and the second command assigns an IP address to the \texttt{tun0}
interface and then activates it. It should be noted that the first
command will block and wait for connections,
so we need to find another window run the second command.


\begin{lstlisting}
$ sudo ./vpnserver

Run the following command in another window:
$ sudo ifconfig tun0 192.168.53.1/24 up
\end{lstlisting}

Unless specifically configured, a computer will only act as a host,
not as a gateway. The VPN Server needs to forward packets to other destinations,
so it needs to function as a gateway. We need to
enable the IP forwarding for a computer to behave like a gateway.
IP forwarding can be enabled
using the following command:

\begin{lstlisting}
$ sudo sysctl net.ipv4.ip_forward=1
\end{lstlisting}



\paragraph{Paso 2: Iniciar el Cliente VPN.} 
We now run the VPN client program on the Client
VM.  We run the following command on this machine (the first command
will connect to the VPN server program running on {\tt 10.0.2.8}.
This command will block as well, so we need to find another window to
configure the \texttt{tun0} interface created by the VPN client program.
We assign IP address \texttt{192.168.53.5} to the \texttt{tun0} interface~(you
can choose other IP addresses).


\begin{lstlisting}
On VPN Client VM:
$ sudo ./vpnclient 10.0.2.8

Run the following command in a different window
$ sudo ifconfig tun0 192.168.53.5/24 up
\end{lstlisting}



\paragraph{Step 3: Set Up Routing on Client and Server VMs.}
After the above two steps, the tunnel will be established.
Before we can use the tunnel, we need to set up routing
paths on both client and server machines to direct the intended traffic through
the tunnel. 
We can use the \texttt{route} command to add an routing entry. The
following example shows how to route the \texttt{10.20.30.0/24}-bound
packets to the interface \texttt{eth0}.

\begin{lstlisting}
$ sudo route add -net 10.20.30.0/24 eth0
\end{lstlisting}

To bypass firewalls on the Client VM, you need to set up 
routing entries accordingly, so the traffics to the blocked site
will be routed towards the VPN. You need to think about what 
routing entries to add in order to bypass the firewall. 



\paragraph{Paso 4: Setup del NAT en la Máquina Virtual del Servidor.}
When the final destination sends packets back to users, the packet
will be sent to the VPN Server first (think about why and write down your answer 
in the report). The return packet will reach the VPN Server's NAT
adapter first (because the source IPs of all  the outgoing
packets from the Server VM are changed to the NAT's external IP address (which is basically the host computer's IP
address in our setup). Usually, the NAT will replace the destination IP address with the IP
address of the original packet (i.e. \texttt{192.168.53.5} in our case), and give it back to whoever owns
the IP address.  Unfortunately, we have a problem here.


Before the NAT sends out the packet, it needs to know the MAC address of the machine who owns
\texttt{192.168.53.5}, so it sends an ARP request. Our private network is virtual, and 
this IP address belongs to the \texttt{tun0} interface on the VPN Client.  
therefore, \texttt{192.168.53.5} will not receive the ARP request (even if it does, it has no
use). The NAT will then drop the packet, because the recipient does not exist.


The actual recipient should be the VPN Server VM, even though it does not own 
\texttt{192.168.53.5}.  If
we can configure the NAT as a gateway, we can ask the NAT to route the packets for
\texttt{192.168.53.5} to 
the VPN Server, which will eventually deliver the packets through the tunnel to the VPN Client. However,
we have not figured out how to configure the NAT as a gateway in VirtualBox, we did
come up two work-around solutions. One idea is to ``fool'' the NAT to believe that the MAC address of
\texttt{192.168.53.5} is the VPN Server VM's MAC address, so the packet will be delivered to
the VPN Server by the NAT. We can achieve this using an ARP cache poisoning on the NAT, basically telling the
NAT before hand about the MAC address of \texttt{192.168.53.5}.

A better solution to get round the limitation of the NAT is to create another NAT
right on the Server VM, so all packets coming out of the Server VM will have this VM's IP address as their source IP.
To reach the Internet, these packets will go through another NAT, which is provided by
VirtualBox, but since the source IP is the Server VM, this second NAT will have no problem relaying back
the returned packets from the Internet to the Server VM. Using this solution, we do not need to use ARP
cache poisoning to ``fool'' the NAT any more. The following commands can enable the NAT on
the Server VM~(in your case, the name of the \texttt{NAT Network} adapter may not be called 
\texttt{enp0s3}; you just need to find its real name on your VM):

    
\begin{lstlisting}
$ sudo iptables -t nat -A POSTROUTING -j MASQUERADE -o enp0s3
\end{lstlisting}
    


\paragraph{Demostración.}
If you have done the steps above correctly, you should be able to bypass
the firewall. You should show that you can reach the blocked web site from Client VM
via the VPN.  Your solution should
not only work for web traffic, but also for all other traffic. For example, if the blocked
machine runs a \texttt{telnet} server, you should be able to \texttt{telnet} to this blocked server from
Client VM. 

In your lab report, you should provide the evidence to show that your traffic did go through
the VPN tunnel, not through some ``side doors''. The best way to show that is to capture the
network traffic using Wireshark, and describe the path of your packets using the captured
traffic. Without such an evidence, we have no idea whether your success is due to a
mis-configured firewall (i.e. the targeted web site is not blocked at all) or due to your VPN.



% *******************************************
% SECTION
% ******************************************* 
\section{Informe del Laboratorio}

%%%%%%%%%%%%%%%%%%%%%%%%%%%%%%%%%%%%%%%%

Debe enviar un informe de laboratorio detallado, con capturas de pantalla, para describir lo que ha hecho y lo que ha observado.
También debe proporcionar una explicación a las observaciones que sean interesantes o sorprendentes.
Enumere también los fragmentos de código más importantes seguidos de una explicación. No recibirán créditos aquellos fragmentos de códigos que no sean explicados.
%%%%%%%%%%%%%%%%%%%%%%%%%%%%%%%%%%%%%%%%

\section*{Agradecimientos}

Este documento ha sido traducido al Español por Facundo Fontana


\end{document}


