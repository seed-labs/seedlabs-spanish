%%%%%%%%%%%%%%%%%%%%%%%%%%%%%%%%%%%%%%%%%%%%%%%%%%%%%%%%%%%%%%%%%%%%%%
%%  Copyright by Wenliang Du.                                       %%
%%  This work is licensed under the Creative Commons                %%
%%  Attribution-NonCommercial-ShareAlike 4.0 International License. %%
%%  To view a copy of this license, visit                           %%
%%  http://creativecommons.org/licenses/by-nc-sa/4.0/.              %%
%%%%%%%%%%%%%%%%%%%%%%%%%%%%%%%%%%%%%%%%%%%%%%%%%%%%%%%%%%%%%%%%%%%%%%

\newcommand{\commonfolder}{../../common-files}

\documentclass[11pt]{article}

\usepackage[most]{tcolorbox}
\usepackage{times}
\usepackage{epsf}
\usepackage{epsfig}
\usepackage{amsmath, alltt, amssymb, xspace}
\usepackage{wrapfig}
\usepackage{fancyhdr}
\usepackage{url}
\usepackage{verbatim}
\usepackage{fancyvrb}
\usepackage{adjustbox}
\usepackage{listings}
\usepackage{color}
\usepackage{subfigure}
\usepackage{cite}
\usepackage{sidecap}
\usepackage{pifont}
\usepackage{mdframed}
\usepackage{textcomp}
\usepackage{enumitem}
\usepackage{hyperref}


% Horizontal alignment
\topmargin      -0.50in  % distance to headers
\oddsidemargin  0.0in
\evensidemargin 0.0in
\textwidth      6.5in
\textheight     8.9in 

\newcommand{\todo}[1]{
\vspace{0.1in}
\fbox{\parbox{6in}{TODO: #1}}
\vspace{0.1in}
}


\newcommand{\unix}{{\tt Unix}\xspace}
\newcommand{\linux}{{\tt Linux}\xspace}
\newcommand{\minix}{{\tt Minix}\xspace}
\newcommand{\ubuntu}{{\tt Ubuntu}\xspace}
\newcommand{\setuid}{{\tt Set-UID}\xspace}
\newcommand{\openssl} {\texttt{openssl}}


\pagestyle{fancy}
\lhead{\bfseries SEED Labs}
\chead{}
\rhead{\small \thepage}
\lfoot{}
\cfoot{}
\rfoot{}


\definecolor{dkgreen}{rgb}{0,0.6,0}
\definecolor{gray}{rgb}{0.5,0.5,0.5}
\definecolor{mauve}{rgb}{0.58,0,0.82}
\definecolor{lightgray}{gray}{0.90}


\lstset{%
  frame=none,
  language=,
  backgroundcolor=\color{lightgray},
  aboveskip=3mm,
  belowskip=3mm,
  showstringspaces=false,
%  columns=flexible,
  basicstyle={\small\ttfamily},
  numbers=none,
  numberstyle=\tiny\color{gray},
  keywordstyle=\color{blue},
  commentstyle=\color{dkgreen},
  stringstyle=\color{mauve},
  breaklines=true,
  breakatwhitespace=true,
  tabsize=3,
  columns=fullflexible,
  keepspaces=true,
  escapeinside={(*@}{@*)}
}

\newcommand{\newnote}[1]{
\vspace{0.1in}
\noindent
\fbox{\parbox{1.0\textwidth}{\textbf{Note:} #1}}
%\vspace{0.1in}
}


%% Submission
\newcommand{\seedsubmission}{
Debe enviar un informe de laboratorio detallado, con capturas de pantalla, para describir lo que ha hecho y lo que ha observado.
También debe proporcionar una explicación a las observaciones que sean interesantes o sorprendentes.
Enumere también los fragmentos de código más importantes seguidos de una explicación. No recibirán créditos aquellos fragmentos de códigos que no sean explicados.}

%% Book
\newcommand{\seedbook}{\textit{Computer \& Internet Security: A Hands-on Approach}, 2nd
Edition, by Wenliang Du. Para más detalles \url{https://www.handsonsecurity.net}.\xspace}

%% Videos
\newcommand{\seedisvideo}{\textit{Internet Security: A Hands-on Approach},
by Wenliang Du. Para más detalles \url{https://www.handsonsecurity.net/video.html}.\xspace}

\newcommand{\seedcsvideo}{\textit{Computer Security: A Hands-on Approach},
by Wenliang Du. Para más detalles \url{https://www.handsonsecurity.net/video.html}.\xspace}

%% Lab Environment
\newcommand{\seedenvironment}{Este laboratorio ha sido testeado en nuestra imagen pre-compilada de una VM con Ubuntu 16.04, que puede ser descargada del sitio oficial de SEED.\xspace}

\newcommand{\seedenvironmentA}{Este laboratorio ha sido testeado en nuestra imagen pre-compilada de una VM con Ubuntu 16.04, que puede ser descargada del sitio oficial de SEED.\xspace}

\newcommand{\seedenvironmentB}{Este laboratorio ha sido testeado en nuestra imagen pre-compilada de una VM con Ubuntu 20.04, que puede ser descargada del sitio oficial de SEED .\xspace}

\newcommand{\seedenvironmentC}{Este laboratorio ha sido testeado en nuestra imagen pre-compilada de una VM con Ubuntu 20.04, que puede ser descargada del sitio oficial de SEED. Sin embargo, la mayoría de nuestros laboratorios pueden ser realizados en la nube para esto Ud. puede leer nuestra guía que explica como crear una VM de SEED en la nube.\xspace}

\newcommand{\seedenvironmentAB}{
Este laboratorio ha sido testeado en nuestras imagenes pre-compiladas de una VM con Ubuntu 16.04 y otra con Ubuntu 20.04, que pueden ser descargadas del sitio oficial de SEED.\xspace}

\newcommand{\nodependency}{Dado que utilizamos contenedores para configurar el entorno de laboratorio, este laboratorio no depende estrictamente de la VM de SEED. Puede hacer este laboratorio utilizando otras máquinas virtuales, máquinas físicas o máquinas virtuales en la nube.\xspace}

\newcommand{\adddns}{You do need to add the required IP address mapping to
the \texttt{/etc/hosts} file.\xspace}






\newcommand{\seedlabcopyright}[1]{
\vspace{0.1in}
\fbox{\parbox{6in}{\small Copyright \copyright\ {#1}\ \ by Wenliang Du.\\
      Este trabajo se encuentra bajo licencia Creative Commons.
       Attribution-NonCommercial-ShareAlike 4.0 International License.
       Si ud. remezcla, transforma y construye a partir de este material,
       Este aviso de derechos de autor debe dejarse intacto o reproducirse de una manera que sea razonable para el medio en el que se vuelve a publicar el trabajo.
       }}
\vspace{0.1in}
}





\newcommand{\pcap} {\texttt{pcap}\xspace}
\newcommand{\telnet} {\texttt{telnet}\xspace}

\lhead{\bfseries SEED Labs -- Laboratorio de Packet Sniffing y Spoofing}


\begin{document}



\begin{center}
{\LARGE Laboratorio de Packet Sniffing y Spoofing}
\end{center}

\seedlabcopyright{2006 - 2020}


\newcounter{task}
\setcounter{task}{1}
\newcommand{\tasks} {\bf {\noindent (\arabic{task})} \addtocounter{task}{1} \,}

\section{Descripción General}

Los conceptos de Packet sniffing y spoofing, son dos de lo más importantes en el campo de la seguridad de redes; ambos representan las mayores amenazas en las comunicaciones de redes. Poder entender estas amenazas es esencial para comprender las medidas de seguridad en las redes. Existen muchas herramientas para hacer sniffing y spoofing tales como {\tt Wireshark}, {\tt Tcpdump}, {\tt Netwox}, \texttt{Scapy}, etc.
Algunas de ellas son ampliamente usadas por expertos en seguridad como así también por atacantes. Saber usar estas herramientas es algo importante para los estudiantes, pero lo que es más importante para ellos en un curso de seguridad de redes es entender como funcionan esas herramientas, es decir como son implementados el sniffing y spoofing de paquetes en el software.

El objetivo de este laboratorio es doble: El primero es aprender como usar esas herramientas y conocer las tecnologías subyacentes de las mismas.
El segundo será escribir un programa simple para hacer sniffing y spoofing de paquetes y así obtener un conocimiento profundo y detallado de los aspectos técnicos de estos programas.
Este laboratorio cubre los siguientes tópicos:


\begin{itemize}[noitemsep]
\item Como funciona el Packet Sniffing y Spoofing
\item Usando la librería {\tt pcap} y Scapy para hacer Packet sniffing
\item Usando Raw Sockets y Scapy para hacer Packet spoofing 
\item Manipulación de paquetes usando Scapy
\end{itemize}



\paragraph{Lecturas y Videos.}
Para una cobertura más detallada sobre Sniffing y Spoofing puede consultar


\begin{itemize}
\item Capítulo 15 del Libro de SEED, \seedbook

\item Sección 2 del curso de SEED en Udemy, \seedisvideo
\end{itemize}



\paragraph{Entorno de Laboratorio.} \seedenvironmentC



\paragraph{Nota para los intructores.}
En este Laboratorio existen dos Sets de Tareas. El primer set se centra en el uso de herramientas para hacer packet sniffing y spoofing. Esto sólo requiere un conocimiento mínimo en Python; los estudiantes no necesitan tener un background previo de programación en Python.
El segundo set de tareas está pensado para estudiantes de Ciencias de la Computación/Ingeniería. 
Los estudiantes deberán escribir sus propios programas desde cero usando el lenguaje C para llevar a cabo el sniffing y el spoofing de paquetes. De esta forma podrán entender en mayor profundidad como es el funcionamiento de un sniffer y de un spoofer. Para abordar esta tarea, es necesario que los estudiantes tenga un sólido conocimiento en el lenguaje de programación C.
Ambos Sets de Tareas son independientes; los instructores en base al background de programación que tengan sus estudiantes pueden elegir asignar un sólo Set o ambos.



% *******************************************
% SECTION
% ******************************************* 
\section{Configuración del entorno usando Contenedores} 

En este laboratorio, usaremoos dos máquinas conectadas a la misma LAN. Podemos usar dós Máquinas Virtuales o dos Contenedores. 
La Figura \ref{fig:labsetup} describe la configuración del entorno de laboratorio utilizando Contenedores.
Realizaremos todos los ataques desde el Contenedor del atacante, mientras que el otro contenedor será la máquina del usuario.

\begin{figure}[htb]
\begin{center}
\includegraphics[width=0.8\textwidth]{\commonfolder/Figs/OneLan_onehost.pdf}
\end{center}
\caption{Configuración del entorno}
\label{fig:labsetup}
\end{figure}
 

%\begin{lstlisting}[backgroundcolor=]
%                +--------------+      +--------------+
%                |  (attacker)  |      |    (user)    |
%                |   10.9.0.1   |      |   10.9.0.5   |
%                +-----+--------+      +------+-------+
%                      | br-<id>              | eth0
%         10.9.0.0/24  |                      |
%         -------------+----------------------+------------
%
%\end{lstlisting}
 


% -------------------------------------------
% SUBSECTION
% -------------------------------------------
\subsection{Setup del Contenedor y sus Comandos} 

%%%%%%%%%%%%%%%%%%%%%%%%%%%%%%%%%%%%%%%%%%%%
Para empezar a preparar el contenedor, deberá de descargarse el archivo \texttt{Labsetup.zip} ubicado en el laboratorio correspondiente dentro del sitio web oficial dentro de la Máquina Virtual prevista por SEED. Una vez descargado deberá descomprimirlo y entrar dentro de la carpeta \texttt{Labsetup} donde encontrará el archivo \texttt{docker-compose.yml} para setear el entorno de laboratorio. Para una información más detallada sobre el archivo \texttt{Dockerfile} y otros archivos relacionados, puede encontrarla dentro del Manual de Usuario del laboratorio en uso, en el sitio web oficial de SEED.

Si esta es su primera experiencia haciendo el setup del laboratorio usando contenedores es recomendable que lea el manual anteriormente mencionado.

A continuación, se muestran los comandos más usados en Docker y Compose.
Debido a que estos comandos serán usados con mucha frecuencia, hemos creados un conjunto de alias para los mismos, ubicados en del archivo \texttt{.bashrc} dentro de la Máquina Virtual provista por SEED (Ubuntu 20.04)

\begin{lstlisting}
$ docker-compose build  # Build the container image
$ docker-compose up     # Start the container
$ docker-compose down   # Shut down the container

// Aliases for the Compose commands above
$ dcbuild       # Alias for: docker-compose build
$ dcup          # Alias for: docker-compose up
$ dcdown        # Alias for: docker-compose down
\end{lstlisting}


Dado que todos los contenedores estarán corriendo en un segundo plano. Necesitamos correr comandos para interactuar con los mismos, una de las operaciones fundamentales es obtener una shell en el contenedor. 
Para este propósito usaremos \texttt{"docker ps"} para encontrar el ID del contenedor deseado y ejecutaremsos\texttt{"docker exec"} para correr una shell en ese contenedor.
Hemos creado un alias para ello dentro del archivo \texttt{.bashrc}

\begin{lstlisting}
$ dockps        // Alias for: docker ps --format "{{.ID}}  {{.Names}}" 
$ docksh <id>   // Alias for: docker exec -it <id> /bin/bash

// The following example shows how to get a shell inside hostC
$ dockps
b1004832e275  hostA-10.9.0.5
0af4ea7a3e2e  hostB-10.9.0.6
9652715c8e0a  hostC-10.9.0.7

$ docksh 96
root@9652715c8e0a:/#  

// Note: If a docker command requires a container ID, you do not need to 
//       type the entire ID string. Typing the first few characters will 
//       be sufficient, as long as they are unique among all the containers. 
\end{lstlisting}

En caso de problemas configurando el entorno, por favor consulte la sección ``Common Problems'' en el manual ofrecido por SEED. 


%%%%%%%%%%%%%%%%%%%%%%%%%%%%%%%%%%%%%%%%%%%%


% -------------------------------------------
% SUBSECTION
% -------------------------------------------
\subsection{El Contenedor del Atacante}

Para este laboratorio, podemos usar la Máquina Virtual o el Contenedor como la máquina del Atacante. Si hecha un vistazo al archivo de Docker Compose, verá que el contenedor del Atacante está configurado de manera diferente del resto de los contenedores. 
Estas son las diferencias:


\begin{itemize}
\item \textit{Directorio Compartido.} Cuando usemos el contenedor del atacante para lanzar los ataques, necesitamos poner el código de ataque dentro del contenedor.

%%%%%%%%%%%%%%%%%%%%%%%%%%%%%%%%%%%%%%%%%%%%%%%
La edición del código es más conveniente dentro de la Máquina Virtual que dentro del contenedor, ya que podemos usar nuestro editor de texto preferido.
Para que la Máquina Virtual y el contenedor puedan compartir archivos, hemos creado un directorio compartido entre ambos para esto hemos usado \texttt{volumes} de Docker.
Dentro del archivo de Docker Composer, encontrará que se ha agregado esta entrada en algunos de los contenedores.
Esta entrada indica que se montará el diretorio \texttt{./volumes} en la Máquina Host (es decir nuestra Máquina Virtual) y se podrá usar dentro del contenedor.
Escribiremos nuestro código dentro del directorio \texttt{./volumes} (en la Máquina Virtual) y este podrá ser usado en el contenedor.

\begin{lstlisting}
volumes:
       - ./volumes:/volumes
\end{lstlisting}


%%%%%%%%%%%%%%%%%%%%%%%%%%%%%%%%%%%%%%%%%%%%%%%


\item \textit{Modo Host.}
%%%%%%%%%%%%%%%%%%%%%%%%%%%%%%%%%%%%%%%%%%%%%%%
En este laboratio, el atacante va a necesitar sniffear los paquetes, pero correr el programa de sniffing dentro del contenedor del atacante tiene sus inconvenientes, ya que el contenedor está atachado a un switch virtual y sólo podrá ver su propio tráfico y no el del resto de los contenedores. Para solucionar este problema, usaremos el modo \texttt{host} para el contenedor del atacante. Esto permite que el contenedor del atacante vea el tráfico de toda la red. La siguiente entrada es usada para el contenedor del atacante:

\begin{lstlisting}
network_mode: host
\end{lstlisting}

Cuando un contenedor está en modo \texttt{host}, este puede ver todas las interfaces de red de los hosts que la componen, inclusive tiene la misma dirección IP como si fuera el host principal. Básicamente es ponerlo en el mismo espacio de red como si fuera la Máquina Virtual de Host. Sin embargo, el contenedor sigue siendo una máquina diferente.



%%%%%%%%%%%%%%%%%%%%%%%%%%%%%%%%%%%%%%%%%%%%%%%
\end{itemize}


\paragraph{Obteniendo el nombre de la interfaz de red.}
%%%%%%%%%%%%%%%%%%%%%%%%%%%%%%%%%%%%%%%%%%%%%%%
Cuando usamos el archivo de Compose para generar los contenedores para el laboratorio, se crea una nueva red que conecta a la Máquina Virtual y a los contenedores. El prefijo de esta red es \texttt{10.9.0.0/24} y se configura dentro del archivo \texttt{docker-compose.yml}. La dirección IP que se asigna a nuestra Máquina Virtual es \texttt{10.9.0.1}. Vamos a necesitar encontrar el nombre de cada una de las interfaces de red en nuestra Máquina Virtual ya que las utilizaremos en nuestros programas. El nombre de la interfaz es la resultante de la concatenación de \texttt{br-} y el ID de red creado por Docker.
Cuando usamos \texttt{ifconfig} para listar las interfaces de red, veremos algunas de ellas. Observe la dirección IP \texttt{10.9.0.1}.


\begin{lstlisting}
$ ifconfig
(*@\textbf{br-c93733e9f913}@*): flags=4163<UP,BROADCAST,RUNNING,MULTICAST>  mtu 1500
        inet (*@\textbf{10.9.0.1}@*)  netmask 255.255.255.0  broadcast 10.9.0.255
        ...
\end{lstlisting}

Otra forma de obtener el nombre de la interfaz es usar el comando \texttt{"docker network"} y encontrar el id de red por nuestra cuenta (el nombre de la red es \texttt{seed-net})

\begin{lstlisting}
$ docker network ls
NETWORK ID          NAME                DRIVER              SCOPE
a82477ae4e6b        bridge              bridge              local
e99b370eb525        host                host                local
df62c6635eae        none                null                local
(*@\textbf{c93733e9f913}@*)        seed-net            bridge              local
\end{lstlisting}



%%%%%%%%%%%%%%%%%%%%%%%%%%%%%%%%%%%%%%%%%%%%%%%



% *******************************************
% SECTION
% ******************************************* 
\section{Set 1 de Tareas: Usando Scapy para el Sniffing and Spoofing de Paquetes}

Se pueden utilizar muchas herramientas para hacer sniffing y spoofing pero la mayoría ofrece funciones limitadas. Scapy es diferente: puede ser usada no sólo como una herramienta sino también como módulo para construir herramientas de sniffing y spoofing es decir podemos integrar funcionalidades de Scapy en nuestros propios programas. En este Set de Tareas usaremos Scapy.

Para usar Scapy, podemos escribir un programa en Python y ejecutarlo usando este interpréte. Vea el siguiente ejemplo. Debemos correr este programa de Python usando privilegios de root, dado que son necesarios para realizar operaciones de spoofing sobre paquetes.
Al comienzo del programa (Línea \ding{192}), importamos todos los módulos de Scapy.

\begin{lstlisting}
# view mycode.py
#!/usr/bin/env python3

from scapy.all import *    (*@\ding{192}@*)

a = IP()
a.show()

# python3 mycode.py
###[ IP ]###
  version   = 4
  ihl       = None
  ...


// Make mycode.py executable (another way to run python programs)
# chmod a+x mycode.py
# mycode.py 
\end{lstlisting}

Podemos utilizar el modo interactivo de Python y correr nuestro programa línea a línea. Esto puede ser más conveniente si necesitamos cambiar de forma frecuente nuestro código en un experimento.


\begin{lstlisting}
# python3
>>> from scapy.all import *
>>> a = IP()
>>> a.show()
###[ IP ]###
  version   = 4
  ihl       = None
  ...
\end{lstlisting}
 

% -------------------------------------------
% SUBSECTION
% ------------------------------------------- 
\subsection{Tarea 1.1: Sniffing de Paquetes}  

Una de las herramientas más populares para hacer sniffing es Wireshark, el uso de esta herramienta es sencillo y la usaremos durante todo el laboratorio. Sin embargo es difícil usar Wireshark para integrarlo a nuestras aplicaciones y crear módulos con diversas funcionalidades. Para este propósito usaremos Scapy. El objetivo de esta Tarea es aprender a usar Scapy para hacer sniffing de paquetes a través de programas de Python.
A continuación se da un ejemplo:


\begin{lstlisting}
#!/usr/bin/env python3
from scapy.all import *

def print_pkt(pkt):
  pkt.show()

pkt = sniff(iface='br-c93733e9f913', filter='icmp', prn=print_pkt)
\end{lstlisting}

El código mostrado anteriormente, hará sniffing de todos los paquetes en la interfaz \texttt{br-c93733e9f913}. Por favor lea las instrucciones en la sección del Setup del Contenedor para saber como obtener esta interfaz.
Si queremos hacer sniffing en más de una interfaz, podemos usar una lista de python y asignarlas a una variable, en este caso la llamaremos \texttt{iface}.
A continuación se da un ejemplo:

\begin{lstlisting}
iface=['br-c93733e9f913', 'enp0s3']
\end{lstlisting}
 


\paragraph{Tarea 1.1A.} En el programa anterior, por cada paquete capturado, la función \texttt{print\_pkt()} será invocada como un callback; esta función mostrará información detallada del paquete. Ejecute el programa con privilegios de root y demuestre como captura los paquetes. Después de eso corra el programa sin privilegios de root; Describa y explique sus observaciones.
 
\begin{lstlisting}
// Make the program executable 
# chmod a+x sniffer.py

// Run the program with the root privilege
# sniffer.py

// Switch to the "seed" account, and
// run the program without the root privilege
# su seed
$ sniffer.py
\end{lstlisting}


\paragraph{Tarea 1.1B.} Generalmente, cuando hacemos sniffing de paquetes, estamos interesados en ciertos tipos de paquetes y no en todos. Para ello usaremos filtros al hacer el sniffing, estos filtros provistos por Scapy usan la sintáxis BPF (Berkeley Packet Filter); Puede encontrar más información sobre BPF en el manual oficial. Por favor  use los siguientes filtros mencionados a continuación y corra el programa nuevamente (Cada filtro debe ser seteado en forma separada):

\begin{itemize} 
 \item Capturar solamente paquetes ICMP.
 \item Capturar cualquier paquete TCP que provenga de una IP y cuyo puerto destino sea el puerto 23.
  \item Capturar paquetes que vienen o se dirigen hacia una subnet en particular. Puede elegir cualquier subnet como \texttt{128.230.0.0/16}; no debería de elegir la subnet a la cual su Máquina Virtual está atachada.
\end{itemize} 



% -------------------------------------------
% SUBSECTION
% ------------------------------------------- 
\subsection{Tarea 1.2: Spoofing de Paquetes ICMP}

Como cualquier herramienta de spoofing, Scapy nos permite establecer los valores de los campos de un paquete IP . El objetivo de esta Tarea consiste en spoofear paquetes IP con una dirección IP origen arbitraria.
Haremos el spoofing sobre paquetes echo request ICMP , y los enviaremos hacia otra Máquina Virtual en la misma red. Usaremos Wireshark para ver si nuestro paquete es aceptado por la máquina destino. Si este paquete es aceptado, se enviará un paquete ICMP echo reply a la dirección IP de origen.
El código a continuación muestra un ejemplo de como spoofear paquetes ICMP.

\begin{lstlisting}
>>> from scapy.all import *
>>> a = IP()              (*@\ding{192}@*)
>>> a.dst = '10.0.2.3'    (*@\ding{193}@*)
>>> b = ICMP()            (*@\ding{194}@*)
>>> p = a/b               (*@\ding{195}@*)
>>> send(p)               (*@\ding{196}@*)
.
Sent 1 packets.
\end{lstlisting}
 
La línea \ding{192} del código mostrado anteriormente, crea un objeto IP usando la clase IP; para cada campo del encabezado IP la clase define un atributo. Podemos usar \texttt{ls(a)} o \texttt{ls(IP)} para ver todos los atributes en forma clave/valor que están disponibles para esa clase. También se puede usar a.show() y IP.show() con el mismo fin. La línea \ding{193} muetra como setear el campo que corresponde a la IP de destino del objeto IP. Si este campo no recibe ningún valor, se usará un valor por defecto.

\begin{lstlisting}
>>> ls(a)
version    : BitField (4 bits)       = 4               (4)
ihl        : BitField (4 bits)       = None            (None)
tos        : XByteField              = 0               (0)
len        : ShortField              = None            (None)
id         : ShortField              = 1               (1)
flags      : FlagsField (3 bits)     = <Flag 0 ()>     (<Flag 0 ()>)
frag       : BitField (13 bits)      = 0               (0)
ttl        : ByteField               = 64              (64)
proto      : ByteEnumField           = 0               (0)
chksum     : XShortField             = None            (None)
src        : SourceIPField           = '127.0.0.1'     (None)
dst        : DestIPField             = '127.0.0.1'     (None)
options    : PacketListField         = []              ([])
\end{lstlisting}
 

La línea \ding{194} crea un objeto ICMP. El tipo por defecto es un echo request. En la línea \ding{195}, se apilarán \texttt{a} y \texttt{b} para formar un nuevo objeto. Esto realiza a través de la sobrecarga del operador \texttt{/} que implementa la clase IP, este operador no representa una división en este caso; este hará que se agregue  \texttt{b} como campo de payload de \texttt{a}, modificando este campo de forma transparente al programador. Como resultado se obitene un nuevo objeto que representa un paquete ICMP. En la línea \ding{196} se envía este paquete usando \texttt{send()}. Por favor realice los cambios necesarios en el código anterior y demuestre que puede spoofear un paquete ICMP del tipo echo request usando una dirección IP origen arbitraria.


% -------------------------------------------
% SUBSECTION
% ------------------------------------------- 
\subsection{Tarea 1.3: Traceroute} 

El objetivo de esta Tarea es usar Scapy para estimar la distancia, en términos de número de routers entre su Máquina Virtual y un destino que se seleccione. Está función es la que implementa una herramienta como \texttt{traceroute}.
En este tarea, escribiremos nuestra propia herramienta. La idea es sencilla: sólo hay que enviar cualquier tipo de paquete a un destino con el valor del campo Time-To-Live (TTL) seteado en 1 como primer paso. Este paquete será dropeado por el primer router que enviará un 
ICMP error message, avisándonos que se ha superado el time-to-live. Así es como obtenemos la dirección IP de nuestro primer router. El paso siguiente será incrementar el valor del campo time-to-live a 2, repetir el proceso y obtener la dirección IP del segundo router. Repetiremos este proceso hasta que nuestro paquete llegue al destino deseado. Cabe aclarar que en este experimento sólo obtendremos un resultado aproximado, en teoría no todos los paquetes que se envían siguen el mismo camino (pero en la práctica, podrían hacerlo por un breve período de tiempo). El código a continuación muestra el procedimiento:


\begin{lstlisting}
a = IP()
a.dst = '1.2.3.4'
a.ttl = 3
b = ICMP()
send(a/b)
\end{lstlisting}

Si es un programador experimentado de Python, puede escribir su herramienta
para realizar todo el procedimiento de forma automática. Si eres nuevo en Python, puede hacerlo cambiando manualmente el campo TTL en cada ronda de ejecución, y registrar la dirección IP basado su observación de Wireshark. Cualquier tipo de implementación es aceptable siempre y cuando se consiga el objettivo.


% -------------------------------------------
% SUBSECTION
% ------------------------------------------- 
\subsection{Tarea 1.4: Sniffing y luego Spoofing}  

En esta Tarea, combinará técnicas de sniffing y de spoofing para implementar un programa de sniff-and-then-spoof.
Necesitará correr dos máquinas en la missma LAN: la Máquina Virtual y un contenedor de usuario. Desde el contenedor de usuario hará un {\tt ping} a una IP X. Esto generará un paquete ICMP echo request. Si X está vivo, el programa {\tt ping} recibirá un echo reply y mostrará la respuesta. Su programa de sniff-and-then-spoof estará corriendo en la Máquina Virtual, que monitorea la LAN a través del paquet sniffing. Cada vez que vea un ICMP echo request, sin importar cual sea la dirección IP de donde provenga, el programa debería de enviar un echo reply usando la técnica de packet spoofing. Además, sin importar si la máquina X esté viva o no, el programa {\tt ping} siempre recibirá una respuesta que indique que X está viva. Para llevar a cabo esta Tarea necesitará usar Scapy. En su informe de laboratorio deberá poner la evidencia que demuestre que su técnica funciona.

En su experimento, debería de hacer \texttt{ping} a las siguientes direcciones IP del contenedor de usuario.
Reporte su observación y explique los resultados.

\begin{lstlisting}
ping 1.2.3.4     # a non-existing host on the Internet
ping 10.9.0.99   # a non-existing host on the LAN
ping 8.8.8.8     # an existing host on the Internet
\end{lstlisting}


\paragraph{Pista:} Para poder explicar correctamente sus observaciones, debería de entender como funciona el protocolo ARP. También necesita saber un poco sobre routing. El comando a continuación lo ayudará a encontrar el router para un destino específico:

\begin{lstlisting}
ip route get 1.2.3.4 
\end{lstlisting}
 
 


% *******************************************
% SECTION
% ******************************************* 
\section{Set 2 de Tareas: Escribiendo programas para hacer sniffing y spoofing}

Para este conjunto de Tareas, debe compilar código en C en la Máquina Virtual Host y correrlo dentro del contenedor.
s
% -------------------------------------------
% SUBSECTION
% ------------------------------------------- 
\subsection{Tarea 2.1: Escribiendo el programa para hacer Packet Sniffing}

Sniffer programs can be easily written using the \pcap library. With \pcap, 
the task of 
sniffers becomes invoking a simple sequence of procedures
in the \pcap library. At the end of the sequence,
packets will be put in buffer for further processing
as soon as they are captured. All the details 
of packet capturing are handled by the \pcap library.
The SEED book provides a sample code, showing how  to write a simple sniffer program using 
\pcap. We include the sample code in the following (see the book for detailed explanation). 

\begin{lstlisting}
#include <pcap.h>
#include <stdio.h>

/* This function will be invoked by pcap for each captured packet.
   We can process each packet inside the function.  
 */
void got_packet(u_char *args, const struct pcap_pkthdr *header,
        const u_char *packet)
{
   printf("Got a packet\n");
}

int main()
{
  pcap_t *handle;
  char errbuf[PCAP_ERRBUF_SIZE];
  struct bpf_program fp;
  char filter_exp[] = "icmp";
  bpf_u_int32 net;

  // Step 1: Open live pcap session on NIC with name eth3
  //         Students needs to change "eth3" to the name 
  //         found on their own machines (using ifconfig).
  handle = pcap_open_live("eth3", BUFSIZ, 1, 1000, errbuf); 

  // Step 2: Compile filter_exp into BPF psuedo-code
  pcap_compile(handle, &fp, filter_exp, 0, net);            
  if (pcap_setfilter(handle, &fp) !=0) {                   
      pcap_perror(handle, "Error:");
      exit(EXIT_FAILURE);
  }

  // Step 3: Capture packets
  pcap_loop(handle, -1, got_packet, NULL);                  

  pcap_close(handle);   //Close the handle
  return 0;
}


// Note: don't forget to add "-lpcap" to the compilation command.
// For example: gcc -o sniff sniff.c -lpcap
\end{lstlisting}


Tim Carstens has also written a tutorial on how to use 
\pcap library to write a sniffer program. The tutorial is 
available at \url{http://www.tcpdump.org/pcap.htm}.  
 

\paragraph{Tarea 2.1A: Entendiendo como funciona un Sniffer}
In this task, students need to write a sniffer program to 
print out the source and destination IP addresses of each captured 
packet. Students can type in the above code or download the sample code from the 
SEED book's website (\url{https://www.handsonsecurity.net/figurecode.html}). 
Students should provide screenshots as evidences to show that their sniffer
program can run successfully and produces expected 
results. In addition, please answer the following questions:

\begin{itemize}
\item \textbf{Pregunta 1.} Please use your own words to describe the sequence of the 
library calls that are essential for sniffer programs. This 
is meant to be a summary, not detailed explanation like the 
one in the tutorial or book.
 
\item \textbf{Pregunta 2.} Why do you need the root privilege to run a sniffer program? Where
does the program fail if it is executed without the root privilege?


\item \textbf{Pregunta 3.} Please turn on and turn off the promiscuous mode in your sniffer
program. Can you demonstrate the difference when this mode is on and off? Please describe how
you can demonstrate this.
\end{itemize}


\paragraph{Tarea 2.1B: Escribiendo Filtros.}
Please write filter expressions for your sniffer program 
to capture each of the followings. You can find online 
manuals for \pcap filters.
In your lab reports, you need to include screenshots to show
the results after applying each of these filters. 
\begin{itemize}
\item Capture the ICMP packets between two specific hosts.
\item Capture the TCP packets with a destination port number 
      in the range from 10 to 100.  
\end{itemize}



\paragraph{Tarea 2.1C: Sniffing Passwords.}
Please show how you can use your sniffer program to capture the 
password when somebody is using \telnet on the 
network that you are monitoring. You may need to modify
your sniffer code to print out the data part of a captured TCP 
packet (\telnet uses TCP). It is acceptable if you print out the entire data part, and then
manually mark where the password (or part of it) is.




% -------------------------------------------
% SUBSECTION
% ------------------------------------------- 
\subsection{Tarea 2.2: Spoofing}

When a normal user sends out a packet, operating systems
usually do not allow the user to set all the fields in the protocol 
headers (such as TCP, UDP, and IP headers). OSes will
set most of the fields, while only allowing users to 
set a few fields, such as the destination IP address, 
the destination port number, etc.  However, if 
users have the root privilege, they can set any 
arbitrary field in the packet headers. This is 
called packet spoofing, and it can be done through
{\em raw sockets}. 


Raw sockets give programmers the absolute control over the packet 
construction, allowing programmers to construct any arbitrary packet, including 
setting the header fields and the payload. Using raw sockets is 
quite straightforward; it involves four steps: (1) create a raw socket,
(2) set socket option, (3) construct the packet, and (4) send 
out the packet through the raw socket. There are 
many online tutorials that can teach you how to 
use raw sockets in C programming. We have linked some tutorials
to the lab's web page. Please read them, and learn how to 
write a packet spoofing program. We show a simple skeleton of 
such a program. 


\begin{lstlisting}
int sd;
struct sockaddr_in sin;
char buffer[1024]; // You can change the buffer size

/* Create a raw socket with IP protocol. The IPPROTO_RAW parameter
 * tells the sytem that the IP header is already included;
 * this prevents the OS from adding another IP header.  */
sd = socket(AF_INET, SOCK_RAW, IPPROTO_RAW);
if(sd < 0) {
    perror("socket() error"); exit(-1);
}

/* This data structure is needed when sending the packets 
 * using sockets. Normally, we need to fill out several 
 * fields, but for raw sockets, we only need to fill out
 * this one field */
sin.sin_family = AF_INET;

// Here you can construct the IP packet using buffer[]  
//    - construct the IP header ...
//    - construct the TCP/UDP/ICMP header ...
//    - fill in the data part if needed ...
// Note: you should pay attention to the network/host byte order.


/* Send out the IP packet. 
 * ip_len is the actual size of the packet. */  
if(sendto(sd, buffer, ip_len, 0, (struct sockaddr *)&sin, 
              sizeof(sin)) < 0) {
      perror("sendto() error"); exit(-1);
}
\end{lstlisting}



\paragraph{Tarea 2.2A: Escriba un programa de spoofing.}
Please write your own packet spoofing program in C. You need 
to provide evidences (e.g., Wireshark packet trace) to show that your 
program successfully sends out spoofed IP packets.


\paragraph{Tarea 2.2B: Haga Spoof de un ICMP Echo Request.}
Spoof an ICMP echo request packet on behalf of another machine (i.e., 
using another machine's IP address as its source IP address). This packet 
should be sent to a remote machine on the Internet (the machine must be
alive). You should turn on your Wireshark, so if your spoofing is successful, 
you can see the echo reply coming back from the remote machine. 



\paragraph{Preguntas.} Por favor responda las siguientes preguntas.

\begin{itemize}
\item \textbf{Pregunta 4.}
Can you set the IP packet length field to an arbitrary value,
regardless of how big the actual packet is? 


\item \textbf{Pregunta 5.} 
Using the raw socket programming, do you have to calculate the 
checksum for the IP header? 

\item \textbf{Pregunta 6.} 
Why do you need the root privilege to run the programs that 
use raw sockets? Where does the program fail if executed without the root 
privilege?

\end{itemize}
 





% -------------------------------------------
% SUBSECTION
% ------------------------------------------- 
\subsection{Tarea 2.3: Sniff y luego Spoof}

In this task, you will combine the sniffing and spoofing techniques
to implement the following sniff-and-then-spoof program. 
You need two VMs on the same LAN. From VM A, you 
{\tt ping} an IP X. This will generate an ICMP echo 
request packet. If X is alive, the {\tt ping} program will receive 
an echo reply, and print out the response. Your sniff-and-then-spoof
program runs on VM B, which monitors the LAN through packet sniffing. Whenever it 
sees an ICMP echo request, regardless of what the target IP address is,
your program should immediately send out an echo reply using the 
packet spoofing technique. Therefore, regardless of whether machine X
is alive or not, the {\tt ping} program will always receive 
a reply, indicating that X is alive. You need to write such a program in C, and
include screenshots in your report to show that 
your program works. Please also attach the code (with adequate amount 
of comments) in your report.







% *******************************************
% SECTION
% ******************************************* 
\section{Guías} 



% -------------------------------------------
% SUBSECTION
% ------------------------------------------- 
\subsection{Llenado de Datos en los Raw Packets}


When you send out a packet using raw sockets, you basically construct 
the packet inside a buffer, so when you need to send it out, you simply
give the operating system the buffer and the size of the packet. 
Working directly on the buffer is not easy, so a common way is to
typecast the buffer (or part of the buffer) into 
structures, such as IP header structure, so you can refer to the elements
of the buffer using the fields of those structures. 
You can define the IP, ICMP, TCP, UDP and other header structures in your 
program. The following example show how you can construct an UDP packet:

 
\begin{lstlisting}
struct ipheader {
   type  field;
   .....
}

struct udpheader {
   type field;
   ......
}

// This buffer will be used to construct raw packet.
char buffer[1024];

// Typecasting the buffer to the IP header structure
struct ipheader *ip = (struct ipheader *) buffer;

// Typecasting the buffer to the UDP header structure
struct udpheader *udp = (struct udpheader *) (buffer
                             + sizeof(struct ipheader));

// Assign value to the IP and UDP header fields.
ip->field = ...;
udp->field = ...;
\end{lstlisting}



% -------------------------------------------
% SUBSECTION
% ------------------------------------------- 
\subsection{Network/Host Byte Order y Conversiones}


You need to pay attention to the network and host byte orders. If you use 
x86 CPU, your host byte order uses {\em Little Endian}, while 
the network byte order uses {\em Big Endian}. Whatever the data you put 
into the packet buffer has to use the network byte order; if you do not 
do that, your packet will not be correct. You actually do not need to worry
about what kind of Endian your machine is using, and you actually should not worry
about if you want your program to be portable. 


What you need to do is to always remember to convert your data to the
network byte order when you place the data into the buffer, and convert
them to the host byte order when you copy the data from the buffer to 
a data structure on your computer. If the data is a single byte, you do not
need to worry about the order, but if the data is a {\tt short}, 
{\tt int}, {\tt long}, or a data type that consists of more than one byte, 
you need to call one of the following functions to convert the data:

\begin{lstlisting}
htonl(): convert unsigned int from host to network byte order.
ntohl(): reverse of htonl().
htons(): convert unsigned short int from host to network byte order.
ntohs(): reverse of htons().
\end{lstlisting}


You may also need to use {\tt inet\_addr()}, {\tt inet\_network()},
{\tt inet\_ntoa()}, {\tt inet\_aton()} to convert 
IP addresses from the dotted decimal form (a string) to a
32-bit integer of network/host byte order. You can get their 
manuals from the Internet.


 

% *******************************************
% SECTION
% ******************************************* 
\section{Informe del Laboratorio}

%%%%%%%%%%%%%%%%%%%%%%%%%%%%%%%%%%%%%%%%

Debe enviar un informe de laboratorio detallado, con capturas de pantalla, para describir lo que ha hecho y lo que ha observado.
También debe proporcionar una explicación a las observaciones que sean interesantes o sorprendentes.
Enumere también los fragmentos de código más importantes seguidos de una explicación. No recibirán créditos aquellos fragmentos de códigos que no sean explicados.
%%%%%%%%%%%%%%%%%%%%%%%%%%%%%%%%%%%%%%%%

% *******************************************
% SECTION
% *******************************************
\section{Agradecimientos}

Este documento ha sido traducido al Español por Facundo Fontana




\end{document}


