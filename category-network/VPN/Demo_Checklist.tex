
\renewcommand{\arraystretch}{1.5}

\begin{longtable}{|p{0.3\textwidth}|p{0.7\textwidth}|}
 \caption{Checklist para la demostración VPN}
 \label{vpn:table:checklist}
 \endfirsthead
 \endhead
 \hline\xrowht[()]{10pt}
 \textbf{\Large Requerimientos} & \textbf{\Large Detalles} \\ 
 \hline
 \hline
 \textbf{Estado Inicial} & 
	\vspace*{-0.3cm}
 	\begin{itemize}[topsep=-0.5cm,leftmargin=0.4cm]
 		\item Reinicie las trés Máquinas Virtuales. Empiece la grabación después de que las Máquinas Virtuales sean reiniciadas. Debería de comenzar la demostración inmediatamente después de que hayan sido reiniciadas. Si espera demasiado, tendrá que hacer los reinicios nuevamente.
 		
		\item Escriba en la terminal \texttt{"last reboot; date"}  para mostrar el tiempo de reinicio y el tiempo actual de cada una de las tres máquinas virtuales. La diferencia entre estos dos tiempos no debería de ser más de 5 minutos.

		\item Muestre las tablas de enrutamiento en las tres máquinas virtuales.
	\end{itemize}
 \\ 
 \hline
 
 \textbf{Testeo Pre-Túnel} & 
 	\vspace*{-0.3cm}
 	\begin{itemize}[topsep=-0.5cm,leftmargin=0.4cm]
 		\item Antes de hacer el setup de la VPN, haga un \texttt{ping} al Host \hostv desde el Host \hostu y explique sus observaciones.
	\end{itemize}
 \\ 
 \hline

 \textbf{Creación del Túnel} & 
 	\vspace*{-0.3cm}
 	\begin{itemize}[topsep=-0.5cm,leftmargin=0.4cm]
	   \item Corra su cliente VPN y su servidor VPN.
		\begin{itemize}
		\item Necesita de tipear los passwords para autenticarse en el servidor, el password no debería de ser visible (se descontarán 10 puntos si vemos sus passwords). Puede usar \texttt{getpass()} para lograr este objetivo (para ver el manual use  ``\texttt{man getpass}")
		
		\item Los passwords no pueden ser hardcodeados en su programa. Si lo hace, se le descontarán 50 puntos.
		\end{itemize}

	   \item Realice esta configuración en todas las máquinas virtuales. Aunque puede poner los comandos de configuración en un script, necesita mostrar el script y explicar los comandos de su script.

	   \item  Muestre las tablas de enrutamiento en las tres máquinas virtuals después de la configuración.
	\end{itemize}
 \\ 
 \hline

 \textbf{Testeo del Ping} & 
 	\vspace*{-0.3cm}
 	\begin{itemize}[topsep=-0.5cm,leftmargin=0.4cm]
 		\item En el Host \hostu: realize un \texttt{ping} al Host \hostv.
		\item Use Wireshark para demostrar que su VPN funciona bien.
		\item Muestrenos la prueba de que el túnel está encriptado.
	\end{itemize}
 \\ 
 \hline

 \textbf{Testeo del Telnet} & 
 	\vspace*{-0.3cm}
 	\begin{itemize}[topsep=-0.5cm,leftmargin=0.4cm]
		\item En el Host \hostu: haga \texttt{telnet} al Host \hostv.
		\item Use Wireshark para demostrar que su VPN funciona bien.
	\end{itemize}
 \\ 
 \hline

 \textbf{Testeo del Tunnel-Breaking} & 
 	\vspace*{-0.3cm}
 	\begin{itemize}[topsep=-0.5cm,leftmargin=0.4cm]
		\item On Host \hostu, telnet to Host \hostv. While keeping the telnet connection alive,
		break the VPN tunnel by stopping the vpn client and/or vpn server programs.
		Then type something in the telnet window. Do you see what you type? What
		happens to the TCP connection? Is the connection broken? 

		\item Let us now reconnect the VPN tunnel (do not wait for too long). 
		Run the client and server programs again, and conduct the necessary
		configuration (no need to explain or show commands). Once the tunnel is
		re-established, what is going to happen to the telnet connection? Please
		describe and explain your observation.

	\end{itemize}
 \\ 
 \hline

 \textbf{Testeo de Paquete Grande} & 
 	\vspace*{-0.3cm}
 	\begin{itemize}[topsep=-0.5cm,leftmargin=0.4cm]
	\item Send a large packet (size \textgreater\space 3000) from Host \hostu to Host \hostv. 
	You can use \texttt{"ping -s"} to do that. 

	\item Use Wireshark to describe and explain your observations.
	\end{itemize}
 \\ 
 \hline

 \textbf{Setup de TLS} & 
 	\vspace*{-0.3cm}
 	\begin{itemize}[topsep=-0.5cm,leftmargin=0.4cm]
		\item Show us how you set up your TLS on both client and server sides.
		\item Show us where you place the server certificates and self-signed certificate.
		\item Show us which lines of code load those certificates.
	\end{itemize}
 \\ 
 \hline

 \textbf{Testeo de MITM} & 
 	\vspace*{-0.3cm}
 	\begin{itemize}[topsep=-0.5cm,leftmargin=0.4cm]
		\item Demonstrate that your system can successfully defeat MITM attacks. You
		need to set up a simulated MITM attack, and demonstrate that your client
		program can defeat it.
	\end{itemize}
 \\ 
 \hline

 \textbf{Explicación del Código 1} & 
	Which lines of code are responsible for the following:
 	\vspace*{0.2cm}
 	\begin{itemize}[topsep=-0.5cm,leftmargin=0.4cm]
		\item verifying that the server certificate is valid
		\item verifying that the server is the owner of the certificate
		\item verifying that the server is the intended server
	\end{itemize}
 \\ 
 \hline

 \textbf{Explicación del Código 2} & 
	Which line of code in the client forces TLS handshake to stop if the server certificate
	verification fails?
 \\ 
 \hline

 \textbf{Explicación del Código 3} & 
	Which line(s) of code do the following?
	\vspace*{0.2cm}
 	\begin{itemize}[topsep=-0.5cm,leftmargin=0.4cm]
		\item sending username and password to the server
		\item getting account information from the shadow file
	\end{itemize}
 \\ 
 \hline

 \textbf{Tiempo Final} & 
	Type \texttt{"last reboot; date"} commands to display the time before ending your demo.
 \\ 
 \hline

\end{longtable}
