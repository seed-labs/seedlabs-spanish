%%%%%%%%%%%%%%%%%%%%%%%%%%%%%%%%%%%%%%%%%%%%%%%%%%%%%%%%%%%%%%%%%%%%%%
%%  Copyright by Wenliang Du.                                       %%
%%  This work is licensed under the Creative Commons                %%
%%  Attribution-NonCommercial-ShareAlike 4.0 International License. %%
%%  To view a copy of this license, visit                           %%
%%  http://creativecommons.org/licenses/by-nc-sa/4.0/.              %%
%%%%%%%%%%%%%%%%%%%%%%%%%%%%%%%%%%%%%%%%%%%%%%%%%%%%%%%%%%%%%%%%%%%%%%

\newcommand{\commonfolder}{../../common-files}

\documentclass[11pt]{article}

\usepackage[most]{tcolorbox}
\usepackage{times}
\usepackage{epsf}
\usepackage{epsfig}
\usepackage{amsmath, alltt, amssymb, xspace}
\usepackage{wrapfig}
\usepackage{fancyhdr}
\usepackage{url}
\usepackage{verbatim}
\usepackage{fancyvrb}
\usepackage{adjustbox}
\usepackage{listings}
\usepackage{color}
\usepackage{subfigure}
\usepackage{cite}
\usepackage{sidecap}
\usepackage{pifont}
\usepackage{mdframed}
\usepackage{textcomp}
\usepackage{enumitem}
\usepackage{hyperref}


% Horizontal alignment
\topmargin      -0.50in  % distance to headers
\oddsidemargin  0.0in
\evensidemargin 0.0in
\textwidth      6.5in
\textheight     8.9in 

\newcommand{\todo}[1]{
\vspace{0.1in}
\fbox{\parbox{6in}{TODO: #1}}
\vspace{0.1in}
}


\newcommand{\unix}{{\tt Unix}\xspace}
\newcommand{\linux}{{\tt Linux}\xspace}
\newcommand{\minix}{{\tt Minix}\xspace}
\newcommand{\ubuntu}{{\tt Ubuntu}\xspace}
\newcommand{\setuid}{{\tt Set-UID}\xspace}
\newcommand{\openssl} {\texttt{openssl}}


\pagestyle{fancy}
\lhead{\bfseries SEED Labs}
\chead{}
\rhead{\small \thepage}
\lfoot{}
\cfoot{}
\rfoot{}


\definecolor{dkgreen}{rgb}{0,0.6,0}
\definecolor{gray}{rgb}{0.5,0.5,0.5}
\definecolor{mauve}{rgb}{0.58,0,0.82}
\definecolor{lightgray}{gray}{0.90}


\lstset{%
  frame=none,
  language=,
  backgroundcolor=\color{lightgray},
  aboveskip=3mm,
  belowskip=3mm,
  showstringspaces=false,
%  columns=flexible,
  basicstyle={\small\ttfamily},
  numbers=none,
  numberstyle=\tiny\color{gray},
  keywordstyle=\color{blue},
  commentstyle=\color{dkgreen},
  stringstyle=\color{mauve},
  breaklines=true,
  breakatwhitespace=true,
  tabsize=3,
  columns=fullflexible,
  keepspaces=true,
  escapeinside={(*@}{@*)}
}

\newcommand{\newnote}[1]{
\vspace{0.1in}
\noindent
\fbox{\parbox{1.0\textwidth}{\textbf{Note:} #1}}
%\vspace{0.1in}
}


%% Submission
\newcommand{\seedsubmission}{
Debe enviar un informe de laboratorio detallado, con capturas de pantalla, para describir lo que ha hecho y lo que ha observado.
También debe proporcionar una explicación a las observaciones que sean interesantes o sorprendentes.
Enumere también los fragmentos de código más importantes seguidos de una explicación. No recibirán créditos aquellos fragmentos de códigos que no sean explicados.}

%% Book
\newcommand{\seedbook}{\textit{Computer \& Internet Security: A Hands-on Approach}, 2nd
Edition, by Wenliang Du. Para más detalles \url{https://www.handsonsecurity.net}.\xspace}

%% Videos
\newcommand{\seedisvideo}{\textit{Internet Security: A Hands-on Approach},
by Wenliang Du. Para más detalles \url{https://www.handsonsecurity.net/video.html}.\xspace}

\newcommand{\seedcsvideo}{\textit{Computer Security: A Hands-on Approach},
by Wenliang Du. Para más detalles \url{https://www.handsonsecurity.net/video.html}.\xspace}

%% Lab Environment
\newcommand{\seedenvironment}{Este laboratorio ha sido testeado en nuestra imagen pre-compilada de una VM con Ubuntu 16.04, que puede ser descargada del sitio oficial de SEED.\xspace}

\newcommand{\seedenvironmentA}{Este laboratorio ha sido testeado en nuestra imagen pre-compilada de una VM con Ubuntu 16.04, que puede ser descargada del sitio oficial de SEED.\xspace}

\newcommand{\seedenvironmentB}{Este laboratorio ha sido testeado en nuestra imagen pre-compilada de una VM con Ubuntu 20.04, que puede ser descargada del sitio oficial de SEED .\xspace}

\newcommand{\seedenvironmentC}{Este laboratorio ha sido testeado en nuestra imagen pre-compilada de una VM con Ubuntu 20.04, que puede ser descargada del sitio oficial de SEED. Sin embargo, la mayoría de nuestros laboratorios pueden ser realizados en la nube para esto Ud. puede leer nuestra guía que explica como crear una VM de SEED en la nube.\xspace}

\newcommand{\seedenvironmentAB}{
Este laboratorio ha sido testeado en nuestras imagenes pre-compiladas de una VM con Ubuntu 16.04 y otra con Ubuntu 20.04, que pueden ser descargadas del sitio oficial de SEED.\xspace}

\newcommand{\nodependency}{Dado que utilizamos contenedores para configurar el entorno de laboratorio, este laboratorio no depende estrictamente de la VM de SEED. Puede hacer este laboratorio utilizando otras máquinas virtuales, máquinas físicas o máquinas virtuales en la nube.\xspace}

\newcommand{\adddns}{You do need to add the required IP address mapping to
the \texttt{/etc/hosts} file.\xspace}






\newcommand{\seedlabcopyright}[1]{
\vspace{0.1in}
\fbox{\parbox{6in}{\small Copyright \copyright\ {#1}\ \ by Wenliang Du.\\
      Este trabajo se encuentra bajo licencia Creative Commons.
       Attribution-NonCommercial-ShareAlike 4.0 International License.
       Si ud. remezcla, transforma y construye a partir de este material,
       Este aviso de derechos de autor debe dejarse intacto o reproducirse de una manera que sea razonable para el medio en el que se vuelve a publicar el trabajo.
       }}
\vspace{0.1in}
}






\lhead{\bfseries SEED Labs -- Laboratorio de Variables de Entorno y Programas \setuid }


\begin{document}


\begin{center}
{\LARGE Laboratorio de Variables de Entorno y Programas \setuid }
\end{center}

\seedlabcopyright{2006 - 2016}


% *******************************************
% SECTION
% *******************************************
\section{Descripción General}

El objetivo de este laboratorio es que los estudiantes entiendan como las variables de entorno afectan el comportamiento de un programa y el del sistema. Las variables de entornos son un conjunto de valores dinámicos con un nombre que pueden afectar la forma en que se comportan los procesos que se encuentran corriendo. Son usadas por la mayoría de los sistemas operativos, desde su introducción en Unix en 1979. Aunque las variables de entorno tienen estos impactos en los programas, muchas veces no son del todo comprendidas por los programadores. Por lo tanto si estas variables son usadas y el programador no sabe que estan siendo usadas, el programa puede tener potenciales vulnerabilidades.

En este laboratorio, los estudiantes entenderán como funcionan las variables de entorno, como son propagadas desde un proceso padre a un proceso hijo y como esto afecta al comportamiento de los programas y al sistema. Nos centraremos particularmente en programas \setuid, los programas \setuid usualmente son programas privilegiados.

Este laboratorio cubre los siguientes tópicos:

\begin{itemize}[noitemsep]
\item Variables de Entorno
\item Programas \setuid
\item Invocar de forma segura programas externos
\item Fugas/Leaking
\item Carga/Linkeo Dinámico
\end{itemize}


\paragraph{Lecturas y Videos.}
Para una cobertura más detallada sobre el mecanismo \setuid, variables de entorno y sus problemas de seguriad puede consultar:

\begin{itemize}
\item Capítulos 1 y 2 del libro de SEED, \seedbook
\item Sección 2 del curso de SEED en Udemy, \seedcsvideo
\end{itemize}

\paragraph{Entorno de Laboratorio.} \seedenvironmentC





% *******************************************
% SECTION
% *******************************************
\section{Tareas del Laborattorio}

Los archivos necesarios para este laboratorio están incluídos en \texttt{Labsetup.zip}, este archivo puede ser descargado del sitio oficial del laboratorio.


% -------------------------------------------
% SUBSECTION
% -------------------------------------------
\subsection{Tarea 1: Manipulando Variables de Entorno}

En esta tarea, estudiaremos los comandoos que pueden ser usados para setear y borrar variables de entorno. Usaremos Bash en la cuenta de seed. La shell por defecto que tiene este usuario esta configurada en el archivo {\tt /etc/passwd}. Puede usar otra shell usando el comando {\tt chsh} (por favor no lo haga para este laboratorio). Por favor realize las siguientes tareas:

\begin{itemize}
\item  Use el comando {\tt printenv} o {\tt env} para imprimir en pantalla las variables de entorno. Si está interesado en alguna en particular, comoo {\tt PWD}, puede usar  {\tt "printenv PWD"} o  {\tt "env | grep PWD"}.

\item Use {\tt export} y {\tt unset} para setear o borrar variables de entorno. Estos comandos, son comandos internos de la shell Bash por lo que no estarán disponibles si usa otro tipo de shell.
\end{itemize}


% -------------------------------------------
% SUBSECTION
% -------------------------------------------
\subsection{Tarea 2: Pasando Variables de Entorno de un Proceso Padre a un Proceso Hijo}

En esta tarea, estudiaremos como un proceso hijo recive las variables de entorno desde su proceso padre. En Unix existe una llamada al sistema llamada {\tt fork()} que se encarga de crear nuevos procesos. El nuevo proceso, conocido como proceso hijo es una copia exacta del proceso que invoca a la llamada al sistema  {\tt fork()} este proceso invocador es conocido como el proceso padre; aunque el proceso hijo sea una copia exacta del proceso padre, hay muchas cosas que no son heredadas por el hijo (para más información sobre esto consulte el manual de  {\tt fork()} tipeando el siguiente comando: {\tt man fork}). En esta tarea, queremos saber como si variables de entorno del padre son heredadas por el hijo o no.


\paragraph{Paso 1.} Por favor compile el programa, corralo y describa sus observaciones. El programa esta dentro del directorio  \texttt{Labsetup}; puede ser compilado usando el comando \texttt{"gcc myprintenv.c"}, el cual generará un binario llamado \texttt{a.out}. Ejecutelo y salve su salida en un archivo usando 
\texttt{"a.out > archivo"}.

\begin{lstlisting}[language=C, caption={\texttt{myprintenv.c}}]
#include <unistd.h>
#include <stdio.h>
#include <stdlib.h>

extern char **environ;
void printenv()
{
  int i = 0;
  while (environ[i] != NULL) {
     printf("%s\n", environ[i]);
     i++;
  }
}

void main()
{
  pid_t childPid;
  switch(childPid = fork()) {
    case 0:  /* child process */
      printenv();          (*@\ding{192}@*)
      exit(0);
    default:  /* parent process */
      //printenv();        (*@\ding{193}@*)
      exit(0);
  }
}
\end{lstlisting}

\paragraph{Paso 2.} Comente {\tt printenv()} en el case del proceso hijo (Línea \ding{192}), descomente {\tt printenv()} en el case del proceso padre (Línea \ding{193}). Compile y corra el programa nuevamente y describa sus observaciones. Salve la salida del programa en otro archivo.


\paragraph{Paso 3.} Compare la diferencia de los dos archivos usando el comando 
 {\tt diff}. Explique sus conclusiones.



% -------------------------------------------
% SUBSECTION
% -------------------------------------------
\subsection{Tarea 3: Variables de Entorno y {\tt execve()}}

En esta tarea, estudiaremos como las variables de entorno son afectadas cuando se ejecuta un nuevo programa a través de {\tt execve()}. La función {\tt execve()} realiza una llamada al sistema que carga un comando y lo ejecuta; esta función no retorna del llamado. No crea un nuevo proceso; en vez de eso, se llamada a la sección text del proceso, y las secciones data, bss y el stack son sobreescritas por ese programa que se carga. Esencialmente, {\tt execve()} corre un nuevo programa dentro del proceso que lo invoca. Estamos interesados en las variables de entorno; Son heredadas por el nuevo programa?


\paragraph{Paso 1.} Por favor compile y corra el siguiente programa, describa sus observaciones. Este programa ejecuta un programa llamado \texttt{/usr/bin/env}, que muestra las variables de entorno del proceso en curso.

\begin{lstlisting}[language=C, caption=\texttt{myenv.c}]
#include <unistd.h>

extern char **environ;
int main()
{
  char *argv[2];

  argv[0] = "/usr/bin/env";
  argv[1] = NULL;
  execve("/usr/bin/env", argv, NULL);    (*@\ding{192}@*)

  return 0 ;
}
\end{lstlisting}


\paragraph{Paso 2.} Cambie la invocación a la función {\tt execve()} (Línea \ding{192}) con la siguiente declaración; describa sus observaciones.


\begin{lstlisting}[language=C]
execve("/usr/bin/env", argv, environ);
\end{lstlisting}


\paragraph{Paso 3.} Por favor explique como es que el nuevo programa obtiene sus variables de entorno.



% -------------------------------------------
% SUBSECTION
% -------------------------------------------
\subsection{Tarea 4: Variables de entorno y {\tt system()}}

En esta tarea, estudiaremos como las variables de entorno son afectadas cuando se ejecuta un nuevo programa a través de la función {\tt system()}. Esta función se usa para ejecutar un comando pero a diferencia de {\tt execve()} que directamente ejecuta el comando, {\tt system()} ejecuta \texttt{"/bin/sh -c comando"} es decir ejecuta el comando  a través de la ejecución de {\tt /bin/sh}.

Si observa la implementación de la función {\tt systte(()(}, podrá ver que usa {\tt execl()} para ejecutar {\tt /bin/sh}; {\tt execl()} llama a {\tt execve()}, pasándole un arreglo con las variables de entorno. 
Ademas usando {\tt system()}, las variables de entorno son pasadas del proceso llamador al nuevo programa {\tt /bin/sh}. Por favor compile y corra el siguiente programa para verificarlo.


\begin{lstlisting}[language=C]
#include <stdio.h>
#include <stdlib.h>

int main()
{
  system("/usr/bin/env");
  return 0 ;
}
\end{lstlisting}



% -------------------------------------------
% SUBSECTION
% -------------------------------------------
\subsection{Tarea 5: Variables de Entorno y Programas \setuid }

{\tt Set-UID} is an important security mechanism in Unix operating systems.
When a \setuid program runs, it assumes the owner's privileges. For
example, if the program's owner is root, when anyone runs this
program, the program gains the root's privileges during its execution. 
{\tt Set-UID} allows us to do many interesting things, but since 
it escalates the user's privilege, it is quite risky. Although the
behaviors of \setuid programs are decided by their program logic, not by
users, users can indeed affect the behaviors via environment variables.
To understand how \setuid programs are affected, let us first figure out
whether environment variables are inherited by the \setuid program's
process from the user's process.


\paragraph{Step 1.} Write the following program that can print out all
the environment variables in the current process.

\begin{lstlisting}[language=C]
#include <stdio.h>
#include <stdlib.h>

extern char **environ;
int main()
{
  int i = 0;
  while (environ[i] != NULL) {
    printf("%s\n", environ[i]);
    i++;
  }
}
\end{lstlisting}


\paragraph{Step 2.} Compile the above program, change its ownership to {\tt
root}, and make it a \setuid program.

\begin{lstlisting}
// Asssume the program's name is foo
$ sudo chown root foo
$ sudo chmod 4755 foo
\end{lstlisting}



\paragraph{Step 3.} In your shell (you need to be in a normal user account,
not the {\tt root} account), use the {\tt export} command to set the
following environment variables (they may have already exist):

\begin{itemize}[noitemsep]
\item {\tt PATH}
\item {\tt LD\_LIBRARY\_PATH}
\item {\tt ANY\_NAME} (this is an environment variable defined by you, so
pick whatever name you want).
\end{itemize}


These environment variables are set in the user's shell process.
Now, run the \setuid program from Step 2 in your shell. After you type the
name of the program in your shell, the shell forks a child process,
and uses the child process to run the program. Please check whether all the
environment variables you set in the shell process (parent) get into
the \setuid child process.  Describe your observation. If there are
surprises to you, describe them.



% -------------------------------------------
% SUBSECTION
% -------------------------------------------
\subsection{Tarea 6: La Variable de Entorno PATH y Programas \setuid }

Because of the shell program invoked, calling {\tt system()}
within a \setuid program is quite dangerous. This is because the actual behavior of the
shell program can be affected by environment variables, such
as {\tt PATH}; these environment variables are provided by the user, who
may be malicious.  By changing these variables, malicious users can control
the behavior of the {\tt Set-UID} program. In {\tt Bash}, you can
change the {\tt PATH} environment variable in the following way (this example
adds the directory {\tt /home/seed} to the beginning of the {\tt PATH} environment variable):


\begin{lstlisting}
$ export PATH=/home/seed:$PATH
\end{lstlisting}



The \setuid program below is supposed to execute the {\tt /bin/ls} command;
however, the programmer only uses the relative path for the {\tt ls}
command, rather than the absolute path:

\begin{lstlisting}[language=C]
int main()
{
  system("ls");
  return 0;
}
\end{lstlisting}

Please compile the above program, change its owner to {\tt root}, and
make it a \setuid program.  Can you get this \setuid program to
run your own malicious code, instead of
{\tt /bin/ls}?  If you can, is your malicious code running with the root privilege?
Describe and explain your observations. \\


\paragraph{Note:}
The \texttt{system(cmd)} function executes the \texttt{/bin/sh} program first, and then
asks this shell program to run the \texttt{cmd} command.
In Ubuntu 20.04 (and several versions before), \texttt{/bin/sh} is actually
a symbolic link pointing to \texttt{/bin/dash}.
This shell program has a countermeasure that prevents itself
from being executed in a \setuid process.  Basically, if \texttt{dash} detects that it is
executed in a \setuid process, it immediately changes the effective user ID to the process's
real user ID, essentially dropping the privilege.


Since our victim program is a \setuid program,
the countermeasure in \texttt{/bin/dash} can prevent our attack.
To see how our attack works without such a countermeasure,
we will link \texttt{/bin/sh} to another shell that does not
have such a countermeasure.  We have installed a shell program
called \texttt{zsh} in our Ubuntu 20.04 VM. We use the following
commands to link \texttt{/bin/sh} to \texttt{/bin/zsh}:

\begin{lstlisting}
$ sudo ln -sf /bin/zsh /bin/sh
\end{lstlisting}




% -------------------------------------------
% SUBSECTION
% -------------------------------------------
\subsection{Task 7: The {\tt LD\_PRELOAD} Environment
Variable and \setuid Programs}


In this task, we study how \setuid programs deal
with some of the environment variables.
Several environment variables, including {\tt LD\_PRELOAD},
{\tt LD\_LIBRARY\_PATH}, and other {\tt LD\_*} influence the
behavior of dynamic loader/linker.
A dynamic loader/linker is the part of an operating system (OS) that
loads (from persistent storage to RAM) and links the shared libraries
needed by an executable at run time.

In Linux, {\tt ld.so} or {\tt ld-linux.so}, are the dynamic
loader/linker (each for different types of binary).
Among the environment variables that affect their behaviors,
{\tt LD\_LIBRARY\_PATH} and {\tt LD\_PRELOAD} are the two
that we are concerned in this lab.
In Linux, {\tt LD\_LIBRARY\_PATH} is a colon-separated set
of directories where libraries should be searched for first, before the
standard set of directories.
{\tt LD\_PRELOAD} specifies a list of additional, user-specified, shared libraries to
be loaded before all others. In this task, we will only
study {\tt LD\_PRELOAD}.


\paragraph{Step 1.}
First, we will see how these environment variables influence the
behavior of dynamic loader/linker when running a normal program.
Please follow these steps:


\begin{enumerate}
  \item Let us build a dynamic link library. Create the following program,
  and name it {\tt mylib.c}. It basically overrides the {\tt sleep()} function
  in {\tt libc}:
\begin{lstlisting}[language=C]
#include <stdio.h>
void sleep (int s)
{
  /* If this is invoked by a privileged program,
     you can do damages here!  */
  printf("I am not sleeping!\n");
}
\end{lstlisting}

  \item We can compile the above program using the following commands (in the
  {\tt -lc} argument, the second character is $\ell$):
\begin{lstlisting}
$ gcc -fPIC -g -c mylib.c
$ gcc -shared -o libmylib.so.1.0.1 mylib.o -lc
\end{lstlisting}



  \item Now, set the {\tt LD\_PRELOAD} environment variable:
\begin{lstlisting}
$ export LD_PRELOAD=./libmylib.so.1.0.1
\end{lstlisting}

  \item Finally, compile the following program {\tt myprog}, and
  in the same directory as the above dynamic link library {\tt
  libmylib.so.1.0.1}:
\begin{lstlisting}[language=C]
/* myprog.c */
#include <unistd.h>
int main()
{
  sleep(1);
  return 0;
}
\end{lstlisting}
\end{enumerate}


\paragraph{Step 2.}
After you have done the above, please run {\tt myprog} under the following
conditions, and observe what happens.

  \begin{itemize}
  \item Make {\tt myprog} a regular program, and run it as a normal user.
  \item Make {\tt myprog} a \setuid root program, and run it as a normal user.
  \item Make {\tt myprog} a \setuid root program, export the {\tt LD\_PRELOAD}
  environment variable again in the root account and run it.

  \item Make {\tt myprog} a \setuid user1 program (i.e., the owner is user1, which
        is another user account), export the {\tt LD\_PRELOAD} environment variable
	again in a different user's account (not-root user) and run it.
  \end{itemize}


\paragraph{Step 3.}
You should be able to observe different behaviors in the scenarios
described above, even though you are running the same program.  You need
to figure out what causes the difference. Environment variables
play a role here. Please design an experiment to figure out the
main causes, and explain why the behaviors in Step 2 are
different. (Hint: the child process
may not inherit the {\tt LD\_*} environment variables).



\subsection{Tarea 8: Invocando programas externos, {\tt system()} vs {\tt execve()}}

Although {\tt system()} and {\tt execve()} can both be used to run new
programs, {\tt system()} is quite dangerous if used in a privileged
program, such as \setuid programs. We have seen how the PATH environment
variable affect the behavior of {\tt system()}, because the variable
affects how the shell works. {\tt execve()} does not have the problem,
because it does not invoke shell. Invoking shell has another dangerous
consequence, and this time, it has nothing to do with environment
variables.  Let us look at the following scenario.


Bob works for an auditing agency, and he needs to investigate a company for a suspected
fraud. For the investigation purpose, Bob needs to be able to read
all the files in the
company's \unix system; on the other hand, to protect the integrity of the system,
Bob should not be able to modify any file.
To achieve this goal, Vince, the superuser of the system,
wrote a special set-root-uid program (see
below), and then gave the executable permission to Bob. This program requires
Bob to type a file name at the command line, and then it will
run {\tt /bin/cat} to display the specified file. Since the program is running
as a root, it can display any file Bob specifies. However, since the program
has no write operations, Vince is very sure that Bob cannot use this special program
to modify any file.

\begin{lstlisting}[language=C, caption=\texttt{catall.c}]
int main(int argc, char *argv[])
{
  char *v[3];
  char *command;

  if(argc < 2) {
    printf("Please type a file name.\n");
    return 1;
  }

  v[0] = "/bin/cat"; v[1] = argv[1]; v[2] = NULL;
  command = malloc(strlen(v[0]) + strlen(v[1]) + 2);
  sprintf(command, "%s %s", v[0], v[1]);

  // Use only one of the followings.
  system(command);
  // execve(v[0], v, NULL);

  return 0 ;
}
\end{lstlisting}


\paragraph{Step 1:} Compile the above program, make it a root-owned
\setuid program. The program will use
{\tt system()} to invoke the command.
If you were Bob, can you compromise the integrity of the system? For example,
can you remove a file that is not writable to you?


\paragraph{Step 2:} Comment out the {\tt system(command)} statement, and
uncomment the {\tt execve()} statement; the program
will use {\tt execve()} to invoke the command. Compile the program, and
make it a root-owned \setuid.
Do your attacks in Step 1 still work? Please describe and explain your observations.



% -------------------------------------------
% SUBSECTION
% -------------------------------------------
\subsection{Tarea 9: Fuga/Leaking}

To follow the Principle of Least Privilege, \setuid programs often
permanently relinquish their root privileges if such privileges are not
needed anymore. Moreover, sometimes, the program needs to hand over its
control to the user; in this case, root privileges must be revoked.
The {\tt setuid()} system call can be used to revoke the privileges.
According to the manual, ``\texttt{setuid()} sets the effective user ID of
the calling process. If the effective UID of the caller is root, the real
UID and saved set-user-ID are also set''. Therefore, if a \setuid program
with effective UID 0 calls \texttt{setuid(n)}, the process will become a
normal process, with all its UIDs being set to \texttt{n}.

When revoking the privilege, one of the common mistakes is capability
leaking. The process may have gained some privileged capabilities when it
was still privileged; when the privilege is downgraded, if the program
does not clean up those capabilities, they may still be accessible by the
non-privileged process. In other words, although the effective user ID of
the process becomes non-privileged, the process is still privileged because
it possesses privileged capabilities.

Compile the following program, change its owner to root, and
make it a \setuid program. Run the program as a normal user.
Can you exploit the capability leaking vulnerability in this program?
The goal is to write to the {\tt /etc/zzz} file as a normal user.

\begin{lstlisting}[language=C, caption=\texttt{cap\_leak.c}]
void main()
{
  int fd;
  char *v[2];

  /* Assume that /etc/zzz is an important system file,
   * and it is owned by root with permission 0644.
   * Before running this program, you should create
   * the file /etc/zzz first. */
  fd = open("/etc/zzz", O_RDWR | O_APPEND);
  if (fd == -1) {
     printf("Cannot open /etc/zzz\n");
     exit(0);
  }

  // Print out the file descriptor value
  printf("fd is %d\n", fd);

  // Permanently disable the privilege by making the
  // effective uid the same as the real uid
  setuid(getuid());

  // Execute /bin/sh
  v[0] = "/bin/sh"; v[1] = 0;
  execve(v[0], v, 0);
}
\end{lstlisting}



% *******************************************
% SECTION
% *******************************************
\section{Informe del Laboratorio}

%%%%%%%%%%%%%%%%%%%%%%%%%%%%%%%%%%%%%%%%

Debe enviar un informe de laboratorio detallado, con capturas de pantalla, para describir lo que ha hecho y lo que ha observado.
También debe proporcionar una explicación a las observaciones que sean interesantes o sorprendentes.
Enumere también los fragmentos de código más importantes seguidos de una explicación. No recibirán créditos aquellos fragmentos de códigos que no sean explicados.
%%%%%%%%%%%%%%%%%%%%%%%%%%%%%%%%%%%%%%%%

% *******************************************
% SECTION
% *******************************************
\section*{Agradecimientos}

Este documento ha sido traducido al Español por Facundo Fontana



\end{document}
