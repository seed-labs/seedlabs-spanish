%%%%%%%%%%%%%%%%%%%%%%%%%%%%%%%%%%%%%%%%%%%%%%%%%%%%%%%%%%%%%%%%%%%%%%
%%  Copyright by Wenliang Du.                                       %%
%%  This work is licensed under the Creative Commons                %%
%%  Attribution-NonCommercial-ShareAlike 4.0 International License. %%
%%  To view a copy of this license, visit                           %%
%%  http://creativecommons.org/licenses/by-nc-sa/4.0/.              %%
%%%%%%%%%%%%%%%%%%%%%%%%%%%%%%%%%%%%%%%%%%%%%%%%%%%%%%%%%%%%%%%%%%%%%%

\newcommand{\commonfolder}{../../common-files}

\documentclass[11pt]{article}

\usepackage[most]{tcolorbox}
\usepackage{times}
\usepackage{epsf}
\usepackage{epsfig}
\usepackage{amsmath, alltt, amssymb, xspace}
\usepackage{wrapfig}
\usepackage{fancyhdr}
\usepackage{url}
\usepackage{verbatim}
\usepackage{fancyvrb}
\usepackage{adjustbox}
\usepackage{listings}
\usepackage{color}
\usepackage{subfigure}
\usepackage{cite}
\usepackage{sidecap}
\usepackage{pifont}
\usepackage{mdframed}
\usepackage{textcomp}
\usepackage{enumitem}
\usepackage{hyperref}


% Horizontal alignment
\topmargin      -0.50in  % distance to headers
\oddsidemargin  0.0in
\evensidemargin 0.0in
\textwidth      6.5in
\textheight     8.9in 

\newcommand{\todo}[1]{
\vspace{0.1in}
\fbox{\parbox{6in}{TODO: #1}}
\vspace{0.1in}
}


\newcommand{\unix}{{\tt Unix}\xspace}
\newcommand{\linux}{{\tt Linux}\xspace}
\newcommand{\minix}{{\tt Minix}\xspace}
\newcommand{\ubuntu}{{\tt Ubuntu}\xspace}
\newcommand{\setuid}{{\tt Set-UID}\xspace}
\newcommand{\openssl} {\texttt{openssl}}


\pagestyle{fancy}
\lhead{\bfseries SEED Labs}
\chead{}
\rhead{\small \thepage}
\lfoot{}
\cfoot{}
\rfoot{}


\definecolor{dkgreen}{rgb}{0,0.6,0}
\definecolor{gray}{rgb}{0.5,0.5,0.5}
\definecolor{mauve}{rgb}{0.58,0,0.82}
\definecolor{lightgray}{gray}{0.90}


\lstset{%
  frame=none,
  language=,
  backgroundcolor=\color{lightgray},
  aboveskip=3mm,
  belowskip=3mm,
  showstringspaces=false,
%  columns=flexible,
  basicstyle={\small\ttfamily},
  numbers=none,
  numberstyle=\tiny\color{gray},
  keywordstyle=\color{blue},
  commentstyle=\color{dkgreen},
  stringstyle=\color{mauve},
  breaklines=true,
  breakatwhitespace=true,
  tabsize=3,
  columns=fullflexible,
  keepspaces=true,
  escapeinside={(*@}{@*)}
}

\newcommand{\newnote}[1]{
\vspace{0.1in}
\noindent
\fbox{\parbox{1.0\textwidth}{\textbf{Note:} #1}}
%\vspace{0.1in}
}


%% Submission
\newcommand{\seedsubmission}{
Debe enviar un informe de laboratorio detallado, con capturas de pantalla, para describir lo que ha hecho y lo que ha observado.
También debe proporcionar una explicación a las observaciones que sean interesantes o sorprendentes.
Enumere también los fragmentos de código más importantes seguidos de una explicación. No recibirán créditos aquellos fragmentos de códigos que no sean explicados.}

%% Book
\newcommand{\seedbook}{\textit{Computer \& Internet Security: A Hands-on Approach}, 2nd
Edition, by Wenliang Du. Para más detalles \url{https://www.handsonsecurity.net}.\xspace}

%% Videos
\newcommand{\seedisvideo}{\textit{Internet Security: A Hands-on Approach},
by Wenliang Du. Para más detalles \url{https://www.handsonsecurity.net/video.html}.\xspace}

\newcommand{\seedcsvideo}{\textit{Computer Security: A Hands-on Approach},
by Wenliang Du. Para más detalles \url{https://www.handsonsecurity.net/video.html}.\xspace}

%% Lab Environment
\newcommand{\seedenvironment}{Este laboratorio ha sido testeado en nuestra imagen pre-compilada de una VM con Ubuntu 16.04, que puede ser descargada del sitio oficial de SEED.\xspace}

\newcommand{\seedenvironmentA}{Este laboratorio ha sido testeado en nuestra imagen pre-compilada de una VM con Ubuntu 16.04, que puede ser descargada del sitio oficial de SEED.\xspace}

\newcommand{\seedenvironmentB}{Este laboratorio ha sido testeado en nuestra imagen pre-compilada de una VM con Ubuntu 20.04, que puede ser descargada del sitio oficial de SEED .\xspace}

\newcommand{\seedenvironmentC}{Este laboratorio ha sido testeado en nuestra imagen pre-compilada de una VM con Ubuntu 20.04, que puede ser descargada del sitio oficial de SEED. Sin embargo, la mayoría de nuestros laboratorios pueden ser realizados en la nube para esto Ud. puede leer nuestra guía que explica como crear una VM de SEED en la nube.\xspace}

\newcommand{\seedenvironmentAB}{
Este laboratorio ha sido testeado en nuestras imagenes pre-compiladas de una VM con Ubuntu 16.04 y otra con Ubuntu 20.04, que pueden ser descargadas del sitio oficial de SEED.\xspace}

\newcommand{\nodependency}{Dado que utilizamos contenedores para configurar el entorno de laboratorio, este laboratorio no depende estrictamente de la VM de SEED. Puede hacer este laboratorio utilizando otras máquinas virtuales, máquinas físicas o máquinas virtuales en la nube.\xspace}

\newcommand{\adddns}{You do need to add the required IP address mapping to
the \texttt{/etc/hosts} file.\xspace}






\newcommand{\seedlabcopyright}[1]{
\vspace{0.1in}
\fbox{\parbox{6in}{\small Copyright \copyright\ {#1}\ \ by Wenliang Du.\\
      Este trabajo se encuentra bajo licencia Creative Commons.
       Attribution-NonCommercial-ShareAlike 4.0 International License.
       Si ud. remezcla, transforma y construye a partir de este material,
       Este aviso de derechos de autor debe dejarse intacto o reproducirse de una manera que sea razonable para el medio en el que se vuelve a publicar el trabajo.
       }}
\vspace{0.1in}
}






\lhead{\bfseries SEED Labs -- Laboratorio de Race Condition}

\def \code#1 {\fbox{\scriptsize{\texttt{#1}}}}

\begin{document}

\begin{center}
{\LARGE Laboratorio de Race Condition}
\end{center}


\seedlabcopyright{2006 - 2020}



% *******************************************
% SECTION
% ******************************************* 
\section{Descripción General}

El objetivo de este laboratorio es que el estudiante gane experiencia en vulnerabilidades de Race Condition, pudiendo pasar de la teoría a la práctica.
Una vulnerabilidad de Race Condition ocurre cuando múltiples procesos quieren acceder y manipular un determinado dato al mismo tiempo, el resultado de la ejecución será indeterminado, ya que estará dado por el orden en que esos procesos realizaron el acceso. Si un programa tiene una vulnerabilidad del tipo Race Condition, un atacante puede correr un conjunto de procesos en paralelo para hacer un ``race'' contra el programa vulnerable, con la intención de alterar el comportamiento del programa.

Para este laboratorio, los estudiantes tendrán disponible un programa con una vulnerabilidad del tipo Race Condition; La tarea será desarrollar un exploit para explotar esta vulnerabilidad y obtener privilegios de root. A su vez cada uno de los estudiantes serán guiados a fin de que puedan atravesar diferentes tipos mitigaciones que son usadas para contrarrestar los ataques de Race Condition.
Los estudiantes necesitarán evaluar en que situaciones sus ataques serán exitosos y en cuales no, así también explicando el porque de este resultado. 
Este laboratorio cubre los siguientes tópicos:

\begin{itemize}[noitemsep]
\item Vulnerabilidad de Race condition
\item Protecciones: Sticky symlink
\item Principio del menor privilegio
\end{itemize}




\paragraph{Lecturas y Videos.}
Los ataques de race condition son abordados en más detalle en los siguientes recursos:

\begin{itemize}
\item Capítulo 7 del Libro de SEED, \seedbook
\item Sección 6 del curso de SEED en Udemy, \seedcsvideo
\end{itemize}


\paragraph{Tópicos relacionados.}
Existen tres laboratorios relacionados a Race Condition. Uno es el laboratorio de Dirty COW, que explota una vulnerabilidad de Race Condition dentro del kernel del sistema operativo (El Capítulo 8 del libro de SEED cubre este ataque). Los otros dos son los laboratorios de Meltdown y Spectre (Capítulos 13 y 14 del libro de SEED). Estos últimos explotan una vulnerabilidad de Race Condition dentro del CPU. Estos cuatro laboratorios proveen una cobertura integral del problema de la vulnerabilidad de Race Condition en diferentes niveles: a nivel aplicación, a nivel kernel y finalmente a nivel hardware.


\paragraph{Entorno de Laboratorio.} \seedenvironmentC


% *******************************************
% SECTION
% ******************************************* 
\section{Configuración del Entorno de Laboratorio}

% -------------------------------------------
% SUBSECTION
% ------------------------------------------- 
\subsection{Desactivando las Contramedidas}

Ubuntu viene con mecanismos para proteger a los programas frente a los ataques de Race Condition. Uno de ellos funciona restringiendo el acceso a los links simbólicos dentro de  sticky directories (por ejemplo {\tt /tmp}) permitiendo el acceso solamente al creador del link si el directorio no fue creado por quién creo el link simbólico.
Otro mecanismo de seguridad que fue implementado en Ubuntu 20.04, evita que el usuario root escriba archivos  dentro del directorio \texttt{/tmp}. que no fueron creados por este usuario.
Como primer paso necesitaremos desactivar estas protecciones, ejecutando los siguientes comandos:

\begin{lstlisting}
// On Ubuntu 20.04, use the following:
$ sudo sysctl -w fs.protected_symlinks=0
$ sudo sysctl fs.protected_regular=0

// On Ubuntu 16.04, use the following:
$ sudo sysctl -w fs.protected_symlinks=0

// On Ubuntu 12.04, use the following:
$ sudo sysctl -w kernel.yama.protected_sticky_symlinks=0
\end{lstlisting}


% -------------------------------------------
% SUBSECTION
% ------------------------------------------- 
\subsection{El Programa Vulnerable}

El siguiente programa es en apariencia inofensivo. Pero contiene una vulnerabilidad de Race Condition.

\begin{lstlisting}[caption={The vulnerable program (\texttt{vulp.c})}]
#include <stdio.h>
#include<unistd.h>

int main()
{
   char* fn = "/tmp/XYZ";
   char buffer[60];
   FILE* fp;

   /* get user input */
   scanf("%50s", buffer );

   if(!access(fn, W_OK)){                 (*@\ding{192}@*)
      fp = fopen(fn, "a+");               (*@\ding{193}@*)
      fwrite("\n", sizeof(char), 1, fp);
      fwrite(buffer, sizeof(char), strlen(buffer), fp);
      fclose(fp);
   }
   else printf("No permission \n");
}
\end{lstlisting}

The program above is a root-owned \setuid program;
it appends a string of user input to
the end of a temporary file {\tt /tmp/XYZ}. Since the code runs
with the root privilege, i.e., its effective use ID is zero, it 
can overwrite any file. To prevent itself from accidentally
overwriting other people's file, the program first 
checks whether the real user ID has the access permission to the file
{\tt /tmp/XYZ}; that is the purpose of the {\tt access()} call in Line
\ding{192}.  If the real user ID indeed has 
the right, the program opens the file in Line \ding{193} and append the user
input to the file. 

At first glance the program does not seem to have any problem.
However, there is a race condition vulnerability in this program: due to the 
time window between the check ({\tt access}) and 
the use ({\tt fopen}), there is a possibility that the file used by
{\tt access()} is different from the file used by {\tt fopen()}, even
though they have the same file name {\tt /tmp/XYZ}.  If a malicious 
attacker can somehow makes {\tt /tmp/XYZ} a symbolic link pointing to
a protected file, such as \texttt{/etc/passwd}, inside the time window, 
the attacker can cause the user input
to be appended to \texttt{/etc/passwd}, and can thus gain the root
privilege. The vulnerable program runs with the root privilege, so
it can overwrite any file.


\paragraph{Haciendo \setuid al programa.}
We first compile the above code, and turn its binary into a \setuid program that is owned by the
root. The following commands achieve this goal:

\begin{lstlisting}
$ gcc vulp.c -o vulp
$ sudo chown root vulp
$ sudo chmod 4755 vulp
\end{lstlisting}



% *******************************************
% SECTION
% *******************************************
\section{Task 1: Choosing Our Target}

We would like to exploit the race condition vulnerability in
the program.
We choose to target the password file \texttt{/etc/passwd}, which is not writable by
normal users. By exploiting the vulnerability, we would like to
add a record to the password file, with a goal of
creating a new user account that has the root privilege.
Inside the password file, each user has an entry, which consists of seven fields
separated by colons~(:). The entry for the root user is listed below.

\begin{lstlisting}
root:x:0:0:root:/root:/bin/bash
\end{lstlisting}

For the root user, the third field~(the user ID field) has a value zero. Namely, when the root
user logs in, its process's user ID is set to zero, giving the process the root privilege.
Basically, the power of the root account does not come from its name, but instead from
the user ID field. If we want to create an account with the root privilege,
we just need to put a zero in this field.


Each entry also contains a password field, which is the second field. In the example above, the
field is set to \texttt{"x"}, indicating that the password is stored in another file called
\texttt{/etc/shadow}~(the shadow file). If we follow this example, we have to use the race condition
vulnerability to modify both password and shadow files, which is not
very hard to do. However, there is a simpler solution. Instead of putting \texttt{"x"} in the
password file, we can simply put the password there, so the operating system will not look for
the password from the shadow file.

The password field does not hold the actual password; it holds the one-way hash value of the
password. To get such a value for a given password, we can add a new user in our own system using
the \texttt{adduser} command, and then get the one-way hash value of our password from
the shadow file.  Or we can simply copy the value from the \texttt{seed} user's entry,
because we know its password is \texttt{dees}. Interestingly, there is a magic value
used in \ubuntu live CD for a password-less account, and the magic value is
\texttt{U6aMy0wojraho} (the 6th character is zero, not letter \texttt{O}). If we put this value
in the password field of a user entry, we only need to hit the return key when prompted for
a password.


\paragraph{Task.} To verify whether the magic password works or not, we
manually (as a superuser) add the following entry to the end of the \texttt{/etc/passwd} file. 
Please report whether you can log into the \texttt{test} account without
typing a password, and check whether you have the root privilege.

\begin{lstlisting}
test:U6aMy0wojraho:0:0:test:/root:/bin/bash
\end{lstlisting}

After this task, please remove this entry from the password file. In the
next task, we need to achieve this goal as a normal user. Clearly, we are
not allowed to do that directly to the password file, but we can exploit a
race condition in a privileged program to achieve the same goal.


\paragraph{Warning.}
In the past, some students accidentally emptied the {\tt /etc/passwd} file 
during the attack (this could be caused by some race condition problems
inside the OS kernel). If you lose
the password file, you will not be able to log in again. To avoid this 
trouble, please make a copy of the original password file or take a
snapshot of the VM. This way, you can easily recover from the mishap. 



% *******************************************
% SECTION
% *******************************************
\section{Task 2: Launching the Race Condition Attack}

The goal of this task is to exploit the race condition vulnerability in 
the vulnerable \setuid program listed earlier. The ultimate goal is to gain
the root privilege.  The most critical step of the attack,  making
\texttt{/tmp/XYZ} point to the password file, must occur within
the window between check and use; namely between the \texttt{access} 
and \texttt{fopen} calls in the vulnerable program. 

% -------------------------------------------
% SUBSECTION
% -------------------------------------------
\subsection{Task 2.A: Simulating a Slow Machine} 

Let us pretend that the machine is very slow, and there is a 10-second time window between
the \texttt{access()} and \texttt{fopen()} calls. To simulate that, we 
add a \texttt{sleep(10)} between them. The program will look like the following:

\begin{lstlisting}
if (!access(fn, W_OK)) {
     sleep(10);
     fp = fopen(fn, "a+");
     ...
\end{lstlisting}
 
With this addition, the \texttt{vulp} program (when re-compiled)
will pause and yield control to the operating system for 10 seconds.
Your job is to manually do something, so when the program resumes
after 10 seconds, the program can help you
add a root account to the system. Please 
demonstrate how you would achieve this. 

You won't be able to modify the file name \texttt{/tmp/XYZ}, because 
it is hardcoded in the program, but you can use symbolic links to 
change the meaning of this name. For example, you can make 
\texttt{/tmp/XYZ} a symbolic link to the \texttt{/dev/null} file. 
When you write to \texttt{/tmp/XYZ}, the actual content will be written
to \texttt{/dev/null}. See the following example (the \texttt{"f"} option 
means that if the link exists, remove the old one first): 

\begin{lstlisting}
$ ln -sf /dev/null /tmp/XYZ
$ ls -ld /tmp/XYZ
lrwxrwxrwx 1 seed seed 9 Dec 25 22:20 /tmp/XYZ -> /dev/null
\end{lstlisting}
 
% -------------------------------------------
% SUBSECTION
% ------------------------------------------- 
\subsection{Task 2.B: The Real Attack}

In the previous task, we have kind of ``cheated'' by asking the vulnerable program
to slow down, so we can launch the attack. This is definitely not a real attack.
In this task, we will launch the real attack. 
Before doing this task, make sure that the \texttt{sleep()} statement is removed 
from the \texttt{vulp} program. 


The typical strategy in race condition attacks is to run the attack
program in parallel to the target program, 
hoping to be able to do the critical step within that time window.
Unfortunately, perfect timing is very hard to achieve, so 
the success of attack is only probabilistic.
The probability of a successful attack might be quite low if the window
is small, but we can run the attack many many times. 
We just need to hit the race condition window once.  


\paragraph{Writing the attack program.} In the simulated attack,
we use the \texttt{"ln -s"} command to make/change symbolic links.
Now we need to do it in a program.
We can use {\tt symlink()} in C to create symbolic links.
Since \linux does not allow one to create a link if the link already exists,
we need to delete the old link first.
The following C code snippet shows how to remove a link and then make
{\tt /tmp/XYZ} point to {\tt /etc/passwd}. Please write your attack 
program.

\begin{lstlisting}
unlink("/tmp/XYZ");
symlink("/etc/passwd","/tmp/XYZ");
\end{lstlisting}



\paragraph{Running the vulnerable program and monitoring results.}
Since we need to run the vulnerable program for many
times, we will write a program to automate this process. 
To avoid manually typing an input to the vulnerable program \texttt{vulp}, 
we can use input redirection. Namely, we save our input in a file, and ask
\texttt{vulp} to get the input from this file using \texttt{"vulp <
inputFile"}. We can also use pipe (an example will be given later). 

It may take a while before our attack can successfully modify the
password file, so we need a way to automatically detect whether the attack is
successful or not. There are many ways to do that; an easy way 
is to monitor the timestamp of the file.
The following shell script runs the \texttt{"ls -l"} command, which
outputs several piece of information about a
file, including the last modified time. By comparing the outputs of the
command with the ones produced previously, we can tell
whether the file has been modified or not. 

The following shell script runs the vulnerable program (\texttt{vulp}) in a loop,
with the input given by the \texttt{echo} command (via a pipe).  
You need to decide what should be the actual input. 
If the attack is successful, i.e.,
the \texttt{passwd} is modified, the shell script will stop. 
You do need to be a little bit patient. 
Normally, you should be able to succeed within 5 minutes. 

\begin{lstlisting}
#!/bin/bash

CHECK_FILE="ls -l /etc/passwd"
old=$($CHECK_FILE)
new=$($CHECK_FILE)
while [ "$old" == "$new" ]    (*@\reflectbox{\ding{217}} \textbf{Check if /etc/passwd is modified}@*)
do
   echo "your input" | ./vulp (*@\reflectbox{\ding{217}} \textbf{Run the vulnerable program}@*)
   new=$($CHECK_FILE)
done
echo "STOP... The passwd file has been changed"
\end{lstlisting}

\paragraph{Verifying success}
When your script terminates,
test the success of your exploit by logging in as the test user and
verifying root privileges.  Then terminate the attack program by 
pressing \texttt{Ctrl-C} in the Terminal window in which you started the program.


\paragraph{A Note.}
If after 10 minutes, your attack is still not 
successful, you can stop the attack, and check the ownership
of the \texttt{/tmp/XYZ} file. If the owner of this file
becomes root, manually delete this file, and try your 
attack again, until your attack becomes successful. 
Please document this observation in your lab report. 
In Task 2.C, we will explain the reason and provide
an improved attack method. 



% -------------------------------------------
% SUBSECTION
% ------------------------------------------- 
\subsection{Task 2.C: An Improved Attack Method}

In Task 2.B, if you have done everything correctly, but still could not succeed
in the attack, check the ownership of \texttt{/tmp/XYZ}. You will find out    
that \texttt{/tmp/XYZ}'s owner has become root (normally, it should be \texttt{seed}). 
If this happens, your attack will never succeed, because your attack
program, running with the \texttt{seed} privilege, can no longer remove or
\texttt{unlink()} it.  
This is because the \texttt{/tmp}
folder has a ``sticky'' bit on, meaning that only the owner of the file can
delete the file, even though the folder is world-writable. 


In Task 2.B, we let you use the root's privilege to delete \texttt{/tmp/XYZ}, 
and then try your attack again. The undesirable condition happens randomly,
so by repeating the attack (with the ``help'' from root), you will eventually
succeed in Task 2.B. Obviously, getting help from root is not a real attack. 
We would like to get rid of that, and do it without root's help.


The main reason for that undesirable situation is that 
our attack program has a problem, a race condition problem, the exact problem that
we are trying to exploit in the victim program. How ironic!  
In the past, when we saw that problem, we simply advised students to 
delete the file and try the attack again. Thanks to one of my students, 
who was determined to figure out what the problem was. Because of his effort,
we finally understand why and have an improved solution. 


The main reason for the situation to happen is that the attack program is 
context switched out right after it removes {\tt /tmp/XYZ} (i.e., \texttt{unlink()}),
but before it links the name to another file (i.e., \texttt{symlink()}. 
Remember, the action to remove the existing 
symbolic link and create a new one is not atomic (it involves two 
separate system calls), so if the context switch occurs in the middle
(i.e., right after the removal of {\tt /tmp/XYZ}),
and the target \setuid program gets a chance to run its 
{\tt fopen(fn, "a+")} statement, it will create a new file with root being the owner.
After that, your attack program can no longer make changes to {\tt /tmp/XYZ}.


Basically, using the \texttt{unlink()} and \texttt{symlink()} approach, 
we have a race condition in our attack program. Therefore, while we 
are trying to exploit the race condition in the target program,
the target program may accidentally ``exploit'' the race condition
in our attack program, defeating our attack.  


To solve this problem, we need to make 
\texttt{unlink()} and \texttt{symlink()} atomic. Fortunately, there 
is a system call that allows us to achieve that. More accurately,
it allows us to atomically swap two symbolic links. 
The following program first makes two symbolic links \texttt{/tmp/XYZ}
and \texttt{/tmp/ABC}, and then using the
\texttt{renameat2} system call to atomically switch them. 
This allows us to change what \texttt{/tmp/XYZ} points to
without introducing any race condition. 

\begin{lstlisting}
#define _GNU_SOURCE
  
#include <stdio.h>
#include <unistd.h>
int main()
{
   unsigned int flags = RENAME_EXCHANGE;

   unlink("/tmp/XYZ"); symlink("/dev/null",   "/tmp/XYZ");
   unlink("/tmp/ABC"); symlink("/etc/passwd", "/tmp/ABC");

   renameat2(0, "/tmp/XYZ", 0, "/tmp/ABC", flags);
   return 0;
}
\end{lstlisting}


\paragraph{Tasks.} Please revise your attack program using 
this new strategy, and try your attack again. If everything 
is done correctly, your attack should be able to succeed. 



% *******************************************
% SECTION
% *******************************************
\section{Task 3: Countermeasures}



% -------------------------------------------
% SUBSECTION
% -------------------------------------------
\subsection{Task 3.A: Applying the Principle of Least Privilege}

The fundamental problem of the vulnerable program in this lab is 
the violation of the {\em Principle of Least Privilege}. 
The programmer does understand that the user who runs the program 
might be too powerful, so he/she introduced {\tt access()} to limit the user's 
power. However, this is not the proper approach. A better
approach is to apply the {\em Principle of Least Privilege}; 
namely, if users do not need certain privilege, the privilege
needs to be disabled.

We can use {\tt seteuid{}} system call to temporarily disable
the root privilege, and later enable it if necessary. Please use 
this approach to fix the vulnerability in the program, and then
repeat your attack. Will you be able to succeed? Please report your
observations and provide explanation.



% -------------------------------------------
% SUBSECTION
% ------------------------------------------- 
\subsection{Task 3.B: Using \ubuntu's Built-in Scheme}

Ubuntu 10.10 and later come with a built-in protection scheme against race condition
attacks. In this task, you need to turn the protection back on using the
following commands:

\begin{lstlisting}
// On Ubuntu 16.04 and 20.04, use the following command:
$ sudo sysctl -w fs.protected_symlinks=1

// On Ubuntu 12.04, use the following command:
$ sudo sysctl -w kernel.yama.protected_sticky_symlinks=1
\end{lstlisting}

Conduct your attack after the protection is turned on.  
Please describe your observations. Please also explain
the followings: (1) How does this protection scheme work?
(2) What are the limitations of this scheme?






% *******************************************
% SECTION
% *******************************************
\section{Informe del Laboratorio}

%%%%%%%%%%%%%%%%%%%%%%%%%%%%%%%%%%%%%%%%

Debe enviar un informe de laboratorio detallado, con capturas de pantalla, para describir lo que ha hecho y lo que ha observado.
También debe proporcionar una explicación a las observaciones que sean interesantes o sorprendentes.
Enumere también los fragmentos de código más importantes seguidos de una explicación. No recibirán créditos aquellos fragmentos de códigos que no sean explicados.
%%%%%%%%%%%%%%%%%%%%%%%%%%%%%%%%%%%%%%%%


\end{document}
