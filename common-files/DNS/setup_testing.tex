%%%%%%%%%%%%%%%%%%%%%%%%%%%%%%%%%%%%%%%%%%%%%%%%%%%%%%%%%%%%%%%%%%%%%%
%%  Copyright by Wenliang Du.                                       %%
%%  This work is licensed under the Creative Commons                %%
%%  Attribution-NonCommercial-ShareAlike 4.0 International License. %%
%%  To view a copy of this license, visit                           %%
%%  http://creativecommons.org/licenses/by-nc-sa/4.0/.              %%
%%%%%%%%%%%%%%%%%%%%%%%%%%%%%%%%%%%%%%%%%%%%%%%%%%%%%%%%%%%%%%%%%%%%%%


% -------------------------------------------
% SUBSECTION
% ------------------------------------------- 
\subsection{Testeando la Configuración DNS}

Desde el contenedor del usuario, ejecutaremos una serie de comandos para asegurarnos que la configuración de nuestro laboratorio sea la adecuada. En su informe de laboratorio, por favor documente los resultados de estas pruebas.


\paragraph{Obtener la dirección IP de \texttt{ns.attacker32.com}.}
Cuando ejecutamos el comando \texttt{dig}, el servidor de DNS local forwardeará la consulta hacia el nameserver del atacante, esto se debe a que en el archivo de configuración del servidor DNS del atacante se agregó el \texttt{forward} para la entrada de esta zona. Además, la respuesta debería de venir del archivo de zona (\texttt{attacker32.com.zone}) que se configuró en nameserver del atacante.
Si esto no es lo que ud. obtiene, la configuración que se hizo ha sido errónea. Por favor describa su observación en el informe del laboratorio.

\begin{lstlisting}
$ dig ns.attacker32.com
\end{lstlisting}


\paragraph{Obtener la dirección IP de \texttt{www.example.com}.} 
Dos nameservers se encuentran hosteando el dominio \texttt{example.com}, uno es el nameserver oficial y el otro es el contenedor del atacante. Consultaremos ambos nameservers y veremos que respuesta obtenemos de ellos.
Por favor ejecute los siguientes comando (desde la máquina de usuario) y describa su observación.

\begin{lstlisting}
// Send the query to our local DNS server, which will send the query
// to example.com's official nameserver. 
$ dig www.example.com

// Send the query directly to ns.attacker32.com 
$ dig @ns.attacker32.com www.example.com
\end{lstlisting}
 
Obviamente, nadie va a consultar \texttt{ns.attacker32.com} para obtener la dirección IP de \texttt{www.example.com}; siempre se consultará el nameserver oficial del dominio \texttt{example.com}. El objetivo del ataque de DNS cache poisoning es hacer que la víctima consulte \texttt{ns.attacker32.com} para obtener la dirección IP de \texttt{www.example.com}. Así, si nuestro es exitoso, si corremos el primer comando  \texttt{dig} sin usar la opción \texttt{@}, deberíamos de obtener el resultado falsificado por el atacante, en vez del resultado autentico del nameserver legítimo del dominio  \texttt{www.example.com}.


